% ------------------------------------------------------------------------------
% Conclusão
% ------------------------------------------------------------------------------

\chapter{Conclusão}\label{chap:conclusao}
	
	Escrever um bom capítulo de conclusão em trabalhos acadêmicos envolve sintetizar os principais achados da pesquisa, refletir sobre o significado desses resultados, e sugerir direções futuras. Aqui estão algumas diretrizes para estruturar este capítulo:

	\begin{itemize}
		\item Resumo dos Principais Achados: Comece recapitulando os principais resultados da pesquisa. Destaque como esses resultados atendem aos objetivos do estudo ou respondem às perguntas de pesquisa.
		
		\item Contextualização: Discuta a importância dos resultados no contexto do campo de estudo. Isso inclui como seus achados se alinham ou divergem de estudos anteriores.
		
		\item Reflexão Crítica: Inclua uma autoavaliação da pesquisa, abordando limitações e como elas podem ter afetado os resultados. Isso demonstra integridade acadêmica e compreensão das nuances da pesquisa.
		
		\item Implicações Práticas e Teóricas: Explique as implicações dos seus resultados para a prática, teoria ou política. Isso mostra a relevância e o valor do seu trabalho.
		
		\item Sugestões para Pesquisas Futuras: Baseando-se nas limitações e nos achados da sua pesquisa, sugira áreas para futuras investigações. Isso ajuda a avançar o campo de estudo.
		
		\item Conclusão Final: Termine com uma conclusão forte que reafirme a contribuição do seu trabalho para o campo de estudo. Isso pode incluir uma declaração poderosa sobre o significado dos seus achados ou uma visão para o futuro da área de pesquisa.
		
		\item Produção Bibliográfica: Se aplicável, liste as publicações geradas a partir da sua pesquisa. Isso pode incluir artigos, apresentações em conferências, ou outros materiais acadêmicos.
	\end{itemize}
