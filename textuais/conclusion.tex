% ------------------------------------------------------------------------------
% Conclusion
% ------------------------------------------------------------------------------

\chapter{Conclusion}\label{chap:final}
	
	%Por fim, gostaríamos de fazer algumas considerações finais sobre o trabalho. Um resumo do conteúdo do trabalho é escrito na Seção \ref{chap:final:recap}. Na Seção \ref{chap:final:selfcriticism} é realizada uma auto-crítica da pesquisa feita até aqui, destacando-se os principais problemas. Por fim, os próximos passos, as expectativas sobre a pesquisa e as possíveis publicações produzidas são descritas na Seção \ref{chap:final:future}.
	%Finally, we would like to make some final remarks. A summary of the work's content is written in Section \ref{chap:final:recap}. In Section \ref{chap:final:selfcriticism}, a self-criticism of the research performed so far is carried out, highlighting the main problems. Finally, the next steps, expectations about the research, and possible publications are described in Section \ref{chap:final:future}.
	
	This chapter concludes the thesis. Firstly, a recapitulation of the main topics explored in the thesis is presented in Section \ref{chap:final:recap}. This serves as a concise summary, highlighting the main ideas, proposals, key findings, and their significance within the research context. By revisiting the main research questions and objectives, the recapitulation provides a holistic view of the thesis's scope and accomplishments.
	
	Following the recapitulation, a critical analysis of the research methodology, limitations, and potential biases is presented in Section \ref{chap:final:selfcriticism}. Self-criticism plays a crucial role in research as it allows for a reflective assessment of the strengths and weaknesses of the study. By acknowledging any limitations or areas for improvement, this section contributes to the overall integrity and maturity of the research project.
	
	Next, the chapter discusses the proposal continuity in Section \ref{chap:final:future}. In the continuity proposals, different ideas are described that can be addressed after this project. The topics are developments of subjects that were part of the work, which could receive more attention and could contribute to the literature. Lastly, the chapter concludes with a list of the bibliographic production generated during the project (Section \ref{chap:final:production}). 
	
	\section{Recapitulation}\label{chap:final:recap}
	
		%* Imageamento em Micro-ondas é um problema inverso importante com aplicações em diversas áreas como identificação de falhas em estruturas, imageamento-através-de-parede, detecção de câncer, entre outros.
		%* A partir de medições de campo na frequência de micro-ondas, o interior de regiões inacessíveis é reconstruído por meio de imagens que se baseiam nas propriedades elétricas dos materiais no espaço investigado.
		%* Esta tese se propôs a revisar de uma maneira ampla os aspectos mais importantes na definição matemática do problema.
		%* O Imageamento em Micro-ondas é um Problema Inverso de Espalhamento Eletromagnético. Neste caso, o objetivo é determinar a causa, i.e., um ou mais espalhadores, do efeito observado, i.e., o campo espalhado observado.
		%* A partir das Equações de Maxwell, é possível obter integrais que relacionam o campo espalhado observado no exterior da região investigada com as propriedades eletromagnéticas dentro da região de interesse.
		%* No entanto, essa relação é não-linear uma vez que o campo eletromagnético dentro da região da região de interesse também é desconhecido e também depende do mapeamento das propriedades elétricas do meio. Logo, ambas as variáveis precisam resolvidas simultaneamente.
		%* Este problema inverso é mal-posto, uma vez que, no geral, não existe solução única nem continuidade na relação entre campo espalhado e meio imageado.
		%* Por causa disso, além das fórmulas integrais tracionais, outras fórmulas também são possíveis para amenizar a má-posição do problema.
		%* Além de mal-posto, o grau de não-linearidade do problema cresce a medida que mais espalhadores ou o contraste dos mesmos sobe. O que dificulta ainda mais a solução.
		%* Esta tese se propôs a revisar de maneira ampla todo arcabouço de metodologias numéricas para resolver o problema, incluindo discussões sobre formas de discretizações, aproximações lineares e métodos de regularização de sistema de equações mal-postas.
		%* Na literatura, os métodos números para o problema inverso de espalhamento eletromagnético se dividem em métodos qualitativos e quantitativos. Enquanto os qualitativos se propõem a obter imagens que evidenciam o formato dos espalhadores, os quantitativos estimam as propriedades elétricas além de recuperar suas geometrias.
		%* Dentro os métodos qualitativos, está o Orthogonality Sampling Method (OSM) que, além de recuperar a geometria dos espalhadores, também é capaz de detectar diferentes níveis de contraste.
		%* Os métodos quantitativos podem ser divididos em determinísticos, i.e., que são baseados em sequências passos bem definidos que produzem sempre a mesma saída para uma mesma entrada, e em métodos qualitativos, i.e., que são baseados em processos estocásticos e podem produzir diferentes saídas para uma mesma entrada.
		%* Destacam-se na literatura atualmente várias abordagens empregando técnicas de Aprendizado Profundo de Máquina. Essas técnicas têm ganhado muito atenção por terem o potencial de viabilizar o imageamento em tempo real.
		%* Além disso, uma outra técnica ainda pouco explorada na literatura é a aplicação de Modelos Substitutos para assistir algoritmos para o problema que dependem da simulação do problema direto para avaliação de potenciais soluções. Esse tipo de técnica tem o potencial de reduzir o custo computacional dos algoritmos ao passo que necessita de boa formas de representação das soluções.
		%* Além disso, outra lacuna é uma plataforma integrada para desenvolvimento e testagem de algoritmos para o problema que integrem o conjunto de indicadores de performance já utilizados na literatura. Ainda mais, falta ainda indicadores para a qualidade da recuperação da geometria de objetos além de que os planejamentos experimentais na literatura são mais orientados a estudos de caso, os quais podem ter pouca capacidade de generalização.
		%* Neste sentido, esta tese tem duas proposições principais que levam em consideração o caso bidimensional do problema inverso eletromagnético.
		%* A primeira proposta é aplicação de Modelos Substitutos para assistir Algoritmos Evolutivos e Métodos de Descida. Diferentemente da outra abordagem proposta na literatura, a representação de soluções é baseada nas imagens geradas pelo método qualitativo OSM. A partir da imagem com a informação da geometria e de possíveis níveis de contraste, é possível transformar o problema inverso em um problema de otimização bidimensional cuja as variáveis de decisão é a atribuição de contraste dos espalhadores e o operador de limiarização que poder refinar a geometria e melhorar a qualidade final da imagem obtida. Em contraste com a metodologia já proposta na literatura, a vantagem é que o número menor de variáveis é muito mais facilmente tratado pelos modelos substitutos e são capazes de obter maior precisão. A desvantagem é que a aplicação do método como um todo fica restrito a problemas que o OSM consegue abordar adequadamente. 
		%* Foram propostos 5 formulações possíveis para algoritmos assistidos por modelos substitutos: três baseados em algoritmos evolutivos e dois baseados em algoritmos de descida.
		%* A segunda proposta é a implementação de uma estrutura de desenvolvimento e testagem de algoritmos que amplo suporte para planejamento experimental, quantificação e comparação de desempenho dos algoritmos para o problema. Através de uma estrutura orientada-a-objeto, foi implementado um software que pode realizar estudos de caso bem como benchmarking, o que ainda não é muito abordado na literatura. Através do benchmarking, o desempenho médio de algoritmos num determinado universo de problemas pode ser avaliado com muito mais robustez que nas metodologias mais tradicionais de estudo de caso. A ferramenta possui amplas funções para explorar diferentes características do problema e medir diferentes aspectos das imagens obtidas pelos algoritmos. O software também conta com rotinas de comparações estatística para uma avaliação mais sólida dos resultados.
		%* Nesta tese, foram considerados experimentos computacionais baseados em estudos de caso e de benchmarking.
		%* O planejamento experimental dos estudos de caso considerou quatro problemas que exploram diferentes cenários para o problema, os quais são espalhadores: simples, múltiplos, não-homogêneos e fortes. Nestes estudos, os métodos propostos foram comparados com métodos determinísticos tradicionais.
		%* O planejamento experimental do estudo de benchmarking focou na comparação de desempenho dos algoritmos propostos. Os conjuntos de testes foram planejados de modo a permitir a avaliação do desempenho do algoritmos com o aumentar do nível de contraste de objetos. 
		%* De uma maneira geral, os resultados apontaram que os algoritmos assistidos por modelos substitutos propostos nesta tese tem a capacidade de, nos casos mais simples, obter imagens com qualidades comparáveis ou ligeiramente melhores por um custo computacional ligeiramente mais alto em alguns casos. Nos casos de alto contraste, a metodologia é capaz de reconstruir objetos que os métodos tradicionais não são capazes por um custo computacional razoável. Foi observado também que as próprias limitações do OSM são um fator que restringe a aplicação da metodologia proposta, principalmente quando se tem muitos espalhadores e muitos níveis de contrastes diferentes. Também foi constatado que uma das versões evolutivas da metodologia proposta possuía um melhor desempenho nos indicadores de qualidades da imagem final, ao passo que seu custo computacional era mais elevado. No entanto, uma das versões baseadas em método de descida apresentou resultados comparavelmente bons em termos dos indicadores de qualidade de imagem final e com um custo computacional muito mais baixo.
		%* Os resultados sugerem que a aplicação dos modelos substitutos a partir da transformação em um problema de otimização bidimensional é muito viável uma vez que a função-objetivo fica muito mais fácil de ser predita pelo modelo substituto e que isso permite a aplicação em situações que métodos tradicionais não são capazes de reconstruir as imagens. Se por um lado o baixo número de variáveis é uma das vantagens, por outro lado, as limitações do OSM para alguns tipos de cenário inviabilizam uma aplicação um pouco mais geral da metodologia proposta, como seria viável com a que já é disponível na literatura.
		%* Estes resultados encorajam o posterior desenvolvimento da técnica de aplicação de modelos substitutos para o modelo de modo a encontrar melhores balanços entre a quantidade de variáveis que o modelo substituto deve lidar com a capacidade de generalização para mais cenários de espalhadores.
		
		Microwave imaging is a significant inverse problem with applications in various fields such as defect identification in structures, through-wall imaging, cancer detection, among others. It involves reconstructing the interior of inaccessible regions based on field measurements at microwave frequencies and images derived from the electrical properties of the materials within the investigated space.
		
		This thesis aimed to comprehensively review the mathematical aspects of the problem, recognizing that microwave imaging is an electromagnetic inverse scattering problem. The objective is to determine the cause, i.e., one or more scatterers, of the observed effect, i.e., the scattered field. Maxwell's Equations provide integral equations that relate the observed scattered field outside the region of interest with the unknown electromagnetic field and the mapping of electrical properties inside the region. However, this relationship is nonlinear due to the unknown nature of the electromagnetic field and the mapping of electrical properties, requiring simultaneous resolution of both variables.
		
		The inverse problem is known to be ill-posed, lacking a unique solution or continuity in the relationship between the scattered field and the imaged medium. To address this, various formulas, including traditional integral formulas, have been developed to mitigate the inherent challenges of the problem. The degree of nonlinearity increases with the number of scatterers or their contrast, further complicating the solution process.
		
		The thesis also presented an extensive overview of numerical methodologies to solve the problem, covering topics such as discretization, linear approximations, and regularization methods for ill-posed systems of equations. In the literature, numerical methods for the inverse electromagnetic scattering problem are classified into qualitative and quantitative methods. Qualitative methods focus on reconstructing the shape of scatterers, while quantitative methods aim to estimate the electrical properties in addition to recovering their geometries.
		
		Noteworthy within qualitative methods is the Orthogonality Sampling Method (OSM), capable of recovering scatterer geometry and detecting different contrast levels. On the other hand, quantitative methods can be further categorized as deterministic or stochastic, depending on whether they rely on well-defined sequences of steps or stochastic processes. Deep Learning techniques have gained attention for their potential in enabling real-time imaging. Additionally, the application of Surrogate Models to assist algorithms for the problem, particularly those dependent on forward problem simulations, is an emerging area that holds promise in reducing computational costs while requiring effective representation methods.
		
		The thesis also identified others shortcomings in the field, such as the lack of an integrated platform for algorithm development and testing, the need for standardized performance indicators, and limited generalization capacity in experimental designs. To address these issues, the thesis proposed two main propositions in the two-dimensional case of the electromagnetic inverse problem.
		
		The first proposal involved applying Surrogate Models to assist Evolutionary Algorithms and Descent Methods, utilizing images generated by the OSM qualitative method for representing solutions. Such approach implies in transforming the inverse problem into a two-dimensional optimization problem, allowing for efficient treatment by Surrogate Models and achieving greater precision than similar proposals in the literature \citep{salucci2022learned}. Five formulations for surrogate model-assisted algorithms were proposed, utilizing both evolutionary and descent algorithms.
		
		The second proposal focused on implementing a comprehensive algorithm development and testing structure that supports experimental design, quantification, and performance comparison of algorithms for the problem. This structure included software with object-oriented features capable of conducting case studies and benchmarking, a less explored area in the literature. The software facilitated exploring different problem characteristics, measuring various aspects of obtained images, and performing statistical comparisons to evaluate results more robustly.
		
		The computational experiments conducted in the thesis included both case studies and benchmarking. The case studies covered four problem scenarios: simple, multiple, nonhomogeneous, and strong scatterers. In these studies, the proposed methods were compared against traditional deterministic methods. The benchmarking study focused on evaluating the performance of the proposed algorithms, specifically considering increasing object contrast levels.
		
		The computational experiments conducted in this thesis provided valuable insights into the performance of algorithms assisted by surrogate models for microwave imaging. In general, the results indicated that these proposed algorithms have the capability to obtain images with comparable or slightly better qualities than traditional methods in the simplest cases. However, this improvement comes at a slightly higher computational cost in some instances.
		
		One of the notable findings was that in high-contrast scenarios, the methodology assisted by surrogate models was able to reconstruct objects that traditional methods struggled with, all while maintaining a reasonable computational cost. This suggests that the surrogate models have the potential to address the limitations of traditional techniques and enable the imaging of challenging objects. It was also observed that the limitations of OSM posed restrictions on the application of the proposed methodology, particularly when dealing with numerous scatterers and varying levels of contrast. This highlights the need to consider the limitations of the qualitative method when applying the proposed approach.
		
		In terms of the performance comparison between different algorithm versions, one of the evolutionary versions (SAEA2) demonstrated better image quality indicators but at a higher computational cost. On the other hand, a version based on the descent method (SADM2) showed comparable results in terms of image quality indicators while requiring significantly fewer computational resources. This finding suggests that a descent-based approach may offer a good balance between performance and computational efficiency.
		
		Based on these results, it is evident that the application of surrogate models in the transformation of the inverse problem into a two-dimensional optimization problem is highly feasible. The objective function becomes much easier to predict using surrogate models, enabling the solution of scenarios where traditional techniques struggle to reconstruct images. The advantage of the approach lies in the reduced number of variables, which can be effectively handled by surrogate models. However, it is important to acknowledge that the limitations of OSM in certain scenarios prevent a more general application of the proposed methodology, unlike what is achievable with existing techniques in the literature \citep{salucci2022learned}.
		
		These promising results encourage further development of the technique of applying surrogate models to strike a better balance between the number of variables that the surrogate model must handle and the ability to generalize to more complex scattering scenarios. By refining the application of surrogate models, it may be possible to overcome current limitations and expand the scope of microwave imaging, opening new avenues for improved imaging performance in challenging environments.
		
	\section{Self-Criticism}\label{chap:final:selfcriticism}
	
		%Alguns pontos críticos na condução desta pesquisa podem são comentados a seguir:
		Some critical points in conducting this research are commented below:
		
		%* Um estudo de caso que é muito comum na literatura são os dados obtidos a partir de medições reais pelo Instituto Fresnel. É muito comum que os autores incluam nas suas pesquisas este estudo de caso uma vez que o sucesso nele pode fornecer mais argumentos em prol de uma aplicação real da metodologia proposta. No entanto, isto não foi considerado neste trabalho uma vez o modelo de propagação eletromagnético desses dados é ligeiramente diferente, uma vez que se trata de um outro tipo de campo incidente e de configuração das antenas de medição e transmissão. Isso demandaria toda uma adaptação no modelo que não seria viável a tempo de conclusão deste trabalho.
		%* O métodos determinísticos tradicionais poderiam ser adicionados no estudo de benchmarking, e assim, seria possível uma análise mais profunda de comparação dos métodos propostos com essas metodologias. No entanto, a opção foi por ser mais sucinto, uma vez que mais algoritmos no estudo resultaria num número maior de gráficos e dados para analisar.
		%* A metodologia proposta por Salucci et al. (2022) é a única referência de aplicação de modelos substitutos no problema e citadas diversas vezes para situar a contribuição deste trabalho. No entanto, esta metodologia não foi implementada e seria muito interessante que isto foi realizado para poder comparar melhor a eficiência da escolha pela transformação em um problema de otimização bidimensional e verificar exemplos onde o método proposto neste trabalho não alcançaria resultados satisfatórios enquanto o proposto por Salucci et al. (2022) poderia ser reconstruir adequadamente.
		%* Nos resultados, as formulações SAEAs levaram mais tempo de execução mesmo quando às vezes tinham o mesmo limite de número de avaliações. Isto sugere que as operações evolutivas no algoritmo podem estar tendo um impacto significativo no custo computacional total. Seria necessário investigar mais afundo a implementar e encontrar otimizações que melhorassem o desempenho desses algoritmos neste indicador.
		
		\begin{itemize}
			\item In this work, a case study based on real measurements from the Fresnel Institute \citep{geffrin2005free}, which is commonly used in the literature, was not considered. Although including this case study could provide additional support for the real-world applicability of the proposed methodology, it was not feasible due to the slight differences in the electromagnetic propagation model used in the data. Adapting the model would require substantial time and effort, which was not available during the completion of this work.
			\item Regarding the benchmark study, the option was made to be more succinct and focus on the proposed methods. Therefore, the traditional deterministic methods were not included in the analysis. While including these methods would have allowed for a more comprehensive comparison, it would have increased the number of graphs and data to analyze, which was not the primary objective of this study.
			\item \cite{salucci2022learned} presented the only methodology available in the literature that applied surrogate models to the problem, and this work frequently cites their contribution to contextualize its own. However, the methodology proposed by them was not implemented in this study. It would have been interesting to include their methodology in the comparison to evaluate the efficiency of the transformation into a two-dimensional optimization problem and identify scenarios where their approach may outperform the one proposed in this work.
			\item In the obtained results, it was observed that the formulations based on the SAEAs took longer to run, even when the number of evaluations had the same limit than SADMs. This suggests that the evolutionary operations employed in the algorithm may have a significant impact on the overall computational cost. Further investigation would be necessary to implement optimizations and improve the performance of these algorithms in terms of computational efficiency.
		\end{itemize}
		
	\section{Continuity Proposals}\label{chap:final:future}
		
		%* Melhoramentos na abordagem de construção da imagem inicial
		
			%Uma das necessidades mais evidentes neste trabalho é ampliar o escopo de aplicação o OSM para poder abordar adequadamente cenários mais difíceis. Para isso, seria necessário pesquisar formas de modificar as equações \eqref{eq:3:qualitative:osm:farfield:equation}-\eqref{eq:3:qualitative:osm:nearfield:indicator} com o objetivo de aumentar a robustez do método frente a problemas com maior grau de não-linearidade. Poderiam ser investigadas formas de aplicação das novas equações integrais \citep{bevacqua2021quantitative} ou formas de decomposição de domínio eficientes \citep{zhang2022iterative}. Outra alternativa mais radical seria formas mais eficientes de obter uma imagem inicial controlando o custo computacional. 
		
		%* Um estudo comparativo mais bem elaborado abordando os métodos determinísticos tradicionais
		
			%Tendo em vista a implementação da ferramenta de benchmark, seria possível fazer um planejamento experimental mais bem elaborado afim de descrever quantitativamente o desempenho médio dos métodos determinísticos tradicionais utilizando os indicadores propostos. Este tipo de estudo ainda não está disponível na literatura e poderia ser útil para detectar possíveis diferenças entre esses algoritmos quando uma comparação mais ampla é realizada. 
			
		%* Criação de conjuntos de testes padrão
		
			%Além da possibilidade de gerar conjuntos aleatórios para avaliação do desempenho, também pode ser interessante elaborar um conjunto de testes padrão que possam ser mais utilizados na literatura e que explorem cenários com significados específicos ou gerais. Tal elaboração poderia facilitar comparações em trabalhos futuros de diversos autores que estudam o problema.
		
		%* Criação de conjuntos de teste baseados no valor de DNL
		
			% de criar um conjuntos de testes padrão, também poderia ser interessante elaborar um mecanismo de geração de problemas baseados num valor alvo de DNL. Isto poderia facilitar estudos que se propõem a analisar a variação dos indicadores de performance dos algoritmos com a variação do DNL. Tal investigação poderia envolver a utilização de técnicas de aprendizagem de máquina que fossem capazes de prever o DNL apenas com as características do problema desejado.
		
		%* Limites de aplicação dos métodos
		
			%Uma observação durante os experimentos deste trabalho foi que nem sempre o DNL é suficiente para quantificar o quão difícil é para um algoritmo resolver um determinado tipo de problema. Em outras palavras, às vezes um mesmo algoritmo pode não conseguir resolver problemas de DNL mais baixo do que alguns que são resolvidos. Além disso, seria interessante elaborar uma forma de quantificar os limites de aplicação do algoritmo levando em conta características do problema. Desta maneira, poderia ser mais eficiente a descrição dos limites de funcionamento de cada método.
		
		%* Implementação do problema 3D
			
			%Uma forma de dar continuidade ao trabalho é formular uma definição padrão para o problema tridimensional e implementar uma biblioteca capaz de dar suporte e seja adaptado para desenvolvimento e testagem de algoritmos nesta situação. Além disso, também implementar a metodologia proposta neste trabalho na versão tridimensional.
		
		There are some topics that can be further explored in a future work:
		
		\begin{itemize}
			\item\textbf{Improvements to the initial image approach}: In this work, one of the evident needs is to expand the scope of application of OSM to address more challenging scenarios effectively. This requires modifying equations \eqref{eq:3:qualitative:osm:farfield:equation}-\eqref{eq:3:qualitative:osm:nearfield:indicator} to enhance the method's robustness against highly nonlinear problems. One approach could be exploring new integral equations \citep{bevacqua2021quantitative} or investigating efficient domain decomposition techniques \citep{zhang2022iterative}. Another alternative could be developing more efficient ways to obtain an initial image while controlling the computational cost.
			\item\textbf{A better elaborated comparative study addressing traditional deterministic methods}: The implementation of a benchmark tool opens up possibilities for a more elaborate experimental design to quantitatively describe the average performance of traditional deterministic methods using the proposed indicators. This type of study is currently lacking in the literature and could provide valuable insights into the differences between these algorithms when subjected to a comprehensive comparison.
			\item\textbf{Creation of standard test sets}: Creating a standard test suite would be beneficial to the field, as it would enable researchers to compare their algorithms using well-defined and meaningful scenarios. This standardized approach would facilitate future studies by multiple authors, ensuring consistency and facilitating comparisons.
			\item\textbf{Creation of test sets based on DNL value}: Furthermore, it would be advantageous to design a problem generation mechanism based on a DNL target value. This mechanism would enable studies analyzing the variation of performance indicators of algorithms with varying DNL. Machine learning techniques could be employed to predict the DNL using only the characteristics of the desired problem, further enhancing the efficiency of these investigations.
			\item\textbf{Limits of application of the methods}: During the experiments in this work, it was observed that DNL alone is not always sufficient to quantify the difficulty of solving a particular problem for an algorithm. In some cases, algorithms may struggle to solve problems with lower DNL while being able to solve higher DNL problems. It would be valuable to develop a way to quantify the application limits of an algorithm, taking into account the problem's specific characteristics. This approach would provide a more comprehensive description of the operating limits of each method.
			\item\textbf{Implementation of the 3D model}: To continue this work, formulating a standard definition for the three-dimensional problem and implementing a library capable of supporting and adapting the development and testing of algorithms in this context would be valuable. Additionally, implementing the methodology proposed in this work in its three-dimensional version would further extend its applicability and provide insights into its performance in more complex scenarios.
		\end{itemize}
	
	\section{Bibliographic Production}\label{chap:final:production}
	
		\noindent\textbf{Journals}:
		\vspace{5mm}
		\begin{itemize}
			\item Batista, A. C., Batista, L. S., \& Adriano, R. (2021). A quadratic programming approach for microwave imaging. \textit{IEEE Transactions on Antennas and Propagation, 69(8)}, 4923-4934.
			\item Vargas, J. O., Batista, A. C., Batista, L. S., \& Adriano, R. (2021). On the computational complexity of the conjugate-gradient method for solving inverse scattering problems. \textit{Journal of Electromagnetic Waves and Applications, 35(17)}, 2323-2334.
			\item Batista, A. C., Adriano, R., \& Batista, L. S. (2021). EISPY2D: An Open-Source Python Library for the Development and Comparison of Algorithms in Two-Dimensional Electromagnetic Inverse Scattering Problems. \textit{arXiv preprint arXiv:2111.02185}. \textbf{Under review}.		
		\end{itemize}
		\vspace{5mm}
		\noindent\textbf{Conferences}:
		\vspace{5mm}
		\begin{itemize}
			\item Vargas, J. O., Batista, A. C. ; Batista, L. S., Adriano, R. A Fast Conjugate Gradient Method for Solving Two-Dimensional Electromagnetic Inverse Scattering Problems. \textit{XX Simpósio Brasileiro de Micro-Ondas e Optoeletrônica, 2022, Natal, RN}. Anais do XX Simpósio Brasileiro de Micro-Ondas e Optoeletrônica, 2022. 
		\end{itemize}
		
