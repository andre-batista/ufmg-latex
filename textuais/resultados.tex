% ------------------------------------------------------------------------------
% Resultados
% ------------------------------------------------------------------------------

\chapter{Resultados}\label{chap:resultados}
	
	Escrever um capítulo de resultados em trabalhos acadêmicos é uma etapa crucial, pois comunica as descobertas da pesquisa. Aqui estão algumas sugestões para estruturar e redigir este capítulo de forma eficaz:

	\begin{itemize}
		\item Introdução Breve: Comece com uma introdução curta que reitere os objetivos da pesquisa e explique o que será apresentado no capítulo.
		
		\item Organização Lógica: Estruture o capítulo de forma lógica, geralmente seguindo a ordem das perguntas de pesquisa ou hipóteses. Isso ajuda os leitores a acompanhar facilmente as descobertas.
		
		\item Apresentação Clara dos Dados: Apresente os resultados de maneira clara e concisa. Use tabelas, gráficos e figuras para ilustrar os dados de forma eficaz, garantindo que cada um seja claramente rotulado e acompanhado de uma legenda explicativa.
		
		\item Descrição dos Resultados: Forneça uma descrição textual dos resultados, destacando as descobertas principais. 
		
		\item Referência aos Objetivos e Hipóteses: Faça referência explícita aos objetivos da pesquisa ou hipóteses ao apresentar os resultados, indicando como cada resultado se relaciona com eles.
		
		\item Precisão e Objetividade: Mantenha a precisão e objetividade ao relatar os resultados. Evite usar linguagem emotiva ou fazer inferências sem suporte dos dados.
		
		\item Tratamento de Dados Negativos ou Inesperados: Se houver resultados negativos ou inesperados, inclua-os e ofereça uma breve descrição. Esses resultados podem ser tão informativos quanto os positivos.
		
		\item Uso de Subseções: Divida o capítulo em subseções, se necessário, para manter a organização e facilitar a leitura. Cada subseção pode abordar diferentes aspectos dos resultados.
		
		\item Consistência com Metodologia: Garanta que a apresentação dos resultados seja consistente com a metodologia descrita anteriormente. Isso inclui o uso dos mesmos termos e definições.
		
		\item Sumário dos Resultados: Conclua o capítulo com um sumário dos principais resultados.
	\end{itemize}