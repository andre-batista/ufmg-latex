% ------------------------------------------------------------------------------
% Introduction
% ------------------------------------------------------------------------------

\chapter{Introduction}\label{chap:introduction}

	\pagenumbering{arabic}
	\setcounter{page}{22}

	%Quando uma criança encontra um presente embrulhado numa caixa, pode ser que ela fique curiosa para saber o que é. Ao mesmo tempo, pode ser que ela não queira abrir a caixa para não estragar a surpresa ou não deixar pistas de que descobriu o presente. Então, para ter uma noção do que é o presente, ela balança a caixa e tenta imaginar o conteúdo pelo som de suas batidas no interior do presente. Como se pode ver, ter a noção do que há no interior de espaços sem acessá-los é uma necessidade que pode ser encontrada no cotidiano. Na verdade, isso está presente até antes do nascimento de um ser humano. A saúde do bebê na barriga da mãe é avaliada então pelas imagens do ultrassom.
	When children find a gift wrapped in a box, they might get curious to know what it is. At the same time, it may be that they do not want to open the box not to ruin the surprise, or not to leave clues that the gift has been found. So, to have a notion of what is the gift, they shake softly the box and try to imagine the content by the sound of its beats inside the pack and its weight. As anyone can see, having the notion of what is inside spaces without accessing them is a need that can be found in everyday life. In fact, this is present even before the birth of a human being. For example, baby health in the mother's belly is evaluated by ultrasound images.
	
	%Imagens do interior de espaços podem ser obtidas através de técnicas como o ultrassom, ressonância magnética, raio-x, entre outros. Obter essas imagens pode ser importante em diversas situações: identificar falhas em estruturas, detectar doenças no corpo de pessoas, identificar a composição de um solo e muito mais. Quando a propriedade elétrica do interior é o objetivo da investigação, então ondas eletromagnéticas são utilizadas para penetrarem o espaço e as características do meio são estudadas pela resposta desta propagação. A escolha desse fenômeno pode ser justificada tanto pelo interesse no estudo das propriedades dielétricas do interior do espaço como para detecção de objetos suspeitos dado o contraste dielétrico que podem ter em relação ao meio onde estão. Esta é a ideia do Imageamento em Microondas.
	Images of the interior of spaces can be obtained through techniques such as ultrasound, magnetic resonance, X-ray, among others \citep{morris2012diagnostic}. Getting these images can be important in several situations: to identify defects in structures \citep{benedetti2007multicrack,bozza2008linear,caorsi2004improved}, to detect diseases in the body \citep{fhager2018microwave,nikolova2011microwave}, to identify the composition of the soil \citep{zhang2010microwave,randazzo2021assessment}, through-the-wall imaging \citep{fedeli2017preliminary,dougu2020truncated}, and more. When the electrical property of the interior is the goal of the investigation, then electromagnetic waves are used to penetrate the space and the characteristics of the medium are studied by the response of the scattered waves. The choice of this phenomenon may be justified by both interests in the study of the dielectric properties of the inner space as for the detection of suspected objects given the dielectric contrast that may have in relation to the medium where they are. This is the idea behind Microwave Imaging \citep{pastorino2000microwave}.
	
	%O Imageamento em Microondas é então um Problema de Espalhamento Eletromagnético Inverso. O problema é composto então pelo campo espalhado conhecido (o efeito) e o objetivo é determinar a imagem do espalhador desconhecido (a causa). Pode ser um problema eletromagnético, a formulação matemática do problema é baseada nas Equações de Maxwell. As propriedades também estão baseadas na teoria sobre Problemas Inversos. De fato, o problema de espalhamento, seja acústico ou eletromagnético, é um exemplo clássico de problema inverso. Além disso, o problema é também não linear uma vez que a distribuição do campo no interior da imagem é desconhecida e depende da imagem desconhecida do espalhador.
	Microwave Imaging is then an Electromagnetic Inverse Scattering Problem \citep{chen2017}. The problem is then composed of the known scattered field (the effect) and the goal is to determine the image of the unknown scatterer (the cause). Because it is an electromagnetic problem, the mathematical formulation is based on Maxwell's Equations \citep{jackson1999classical}. The properties are also based on the theory about Inverse Problems \citep{kirsch2011introduction}. In fact, the scattering problem, whether acoustic or electromagnetic, is a classic example of an inverse problem \citep{bertero2020introduction}. In addition, it is also nonlinear since the field distribution inside the image is unknown and depends on the unknown image of the scatterer \citep{colton2019inverse}.
	
	%Soluções para o problema são obtidas computacionalmente. As imagens são reconstruídas por métodos numéricos que buscam resolver as equações que relacionam as propriedades elétricas do interior da região imageada com o campo elétrico observado. Esses métodos podem ser qualitativos ou quantitativos, i.e., fazem reconstruções indiretas ou diretas das propriedades dielétricas. Dentre os quantitativos, existem os determinísticos e os estocásticos. Os determinísticos são aqueles cujas as operações são completamente determinadas e possuem resultados únicos, i.e., sempre irão fazer a mesma reconstrução para uma mesma entrada. Já os estocásticos são métodos cujos passos são influenciados por operações pseudo-aleatórias, e por isso, podem ter diferentes resultados para uma mesma entrada.
	Solutions are computationally obtained \citep{chen2017,pastorino2010}. The images are recovered by numerical methods that attempt to solve the equations that relate the electrical properties of the investigated region with the observed electric field. These methods may be qualitative \citep{pastorino2010qualitative} or quantitative \citep{pastorino2010chap7}, i.e., make indirect or direct reconstructions of dielectric properties. Among the quantitative methods, there are deterministic and stochastic ones. The deterministic are those whose operations are completely determined and have unique results, i.e., they will always yield the same reconstruction for the same input \citep{pastorino2010deterministic}. Stochastics are methods whose steps are influenced by pseudo-random operations, and therefore, they may have different results for the same input \cite{pastorino2007stochastic}. In recent years, deep machine learning techniques have gained attention in the field of microwave imaging due to their potential for real-time imaging and improved accuracy \citep{chen2020review}.
	
	%Dentre os métodos estocásticos, se destacam os Algoritmos Evolucionários. Estes algoritmos são metaheurísticas de população baseada processo biológicos. Em outras palavras, são metódos que utilizam uma população de soluções-candidatas as quais se relacionam por operações matemáticas inspiradas em fenômenos do comportamento de indivíduos numa população ao longo do tempo. Por serem metaheurísticas, elas também não garantem que a solução final para o problema é a melhor existente.
	Among the stochastic methods, the Evolutionary Algorithms stand out \citep{eiben2015introduction}. These algorithms are population-based metaheuristics that imitate biological processes. In other words, they are methods that use a population of candidate solutions and their relationships are mathematical operations inspired by the behavior of individuals in a population over time. Because they are metaheuristics, they also do not guarantee that the final solution is the best existing.
	
	% * A eficiência dos Algoritmos Evolutivos pode diminuir a medida que o número de variáveis do problema começa a crescer.
	% * Além da dificuldade da exploração do espaço de busca, o crescimento do número de variáveis pode impactar a avaliação da função-objetivo tornando-a tão computacionalmente cara que passa ser inviável a aplicação do método.
	% * Uma das formas de contornar tal problema é a utilização de Modelos Substitutos.
	% * Modelos Substitutos são técnicas para aproximar o comportamento de sistemas complexos. Quando aplicados a funções-objetivo em algoritmos de otimização, essas técnicas servem como funções de interpolação as quais são capazes de fazer uma predição da saída para uma dada entrada de uma maneira mais barata computacionalmente.
	% * Desta forma, ao invés de avaliar diretamente a função-objetivo cara, o processo de busca de métodos como os Algoritmos Evolutivos podem ser viabilizados pelo modelo substituto e este ser atualizado com certa frequência.
	
	The efficiency of Evolutionary Algorithms can be significantly affected when dealing with problems that have a large number of variables \citep{gould2005numerical}. As the number of variables increases, the search space becomes larger and more challenging to explore. This, in turn, impacts the evaluation of the objective function, often leading to computationally expensive calculations that hinder the application of EAs \cite{omidvar2022review}.
	
	To address this issue, one promising approach is to incorporate surrogate models into the optimization process. Surrogate Models, also known as metamodels or response surface models, are techniques used to approximate the behavior of complex systems \cite{box1992,krige1951statistical,hardy1972multiquadric,boser1992training}. When applied to objective functions in optimization algorithms, they act as interpolation functions, providing computationally cheaper predictions of the output for a given input \citep{he2023review}. By using surrogate models, the direct evaluation of the expensive objective function can be replaced with evaluations of the surrogate model. This allows for a more efficient search process, as the surrogate model serves as a proxy for the actual objective function. The surrogate model is initially constructed based on a limited number of evaluations of the objective function, and it is then updated or refined as the optimization process progresses.
	
	The key advantage of using surrogate models in optimization algorithms, such as EAs \citep{valadao2020comparative}, is the significant reduction in computational cost. Instead of repeatedly evaluating the expensive objective function, the surrogate model provides quick estimates, enabling faster exploration of the search space. This allows Evolutionary Algorithms to make informed decisions and guide the search towards promising regions of the solution space. However, it is important to note that surrogate models introduce approximation errors, as they are not exact representations of the objective function. The accuracy of the surrogate model depends on factors such as the quality and representativeness of the training data, the chosen surrogate modeling technique, and the underlying assumptions made during the modeling process. Careful validation and refinement of the surrogate model are necessary to ensure its reliability and effectiveness in guiding the optimization process \citep{he2023review,haftka2016parallel}.
	
	
	\section{Motivation}\label{chap:introduction:motivation}
	
		%Na história de aplicação de Algoritmos Evolutivos em Problemas de Espalhamento Eletromagnético Inverso, muitas considerações já foram feitas. Inicialmente, essa metodologias foram aplicadas em situações ondes alguns pressupostos sobre a forma, quantidade ou até as propriedades elétricas dos espalhadores já estavam asseguradas. Posteriormente, outras aplicações surgiram as quais não se baseavam nestas premissas, e por isso, eram mais difícieis de ser resolvidos. A dificuldade deste tipo de aplicação consiste principalmente na quantidade enorme de variáveis envolvidas na modelagem do problema. No entanto, o potencial para abordar problemas não-lineares e a independência de operações computacionais muito caras (como simulações ou decomposições) são vantagem atrativas para o desenvolvimento e aplicação destes métodos.
		In the history of the application of Evolutionary Algorithms in Electromagnetic Inverse Scattering Problems, many studies have already been made \citep{rocca2009evolutionary}. Initially, these methodologies were applied in situations where some assumptions on the shape, amount, or even the electrical properties of the scatterers were already assured \citep{michalski2000electromagnetic,kent1997dielectric}. Subsequently, other applications emerged which were not based on these premises, and therefore, they were more difficult to be solved \citep{chiu1996image,huang2008time}. The difficulty consists mainly in the huge amount of variables involved in the model \citep{caorsi2000two,donelli2006integrated,salucci2017multifrequency}. However, the potential to address nonlinear problems and the independence of expensive computational operations (such as simulations or decompositions) are attractive advantages for the development and application of these methods.
	
		%Na literatura, apenas formulações tradicionais de Algoritmos Evolutivos foram considerados. Existem muitas aplicações de Algoritmos Genéticos, Evolução Diferencial e Otimização de Exame de Partículas. No entanto, na literatura especializada em Algoritmos Evolutivos, já houveram novos avanços e proposições os quais têm o potencial para abordar problemas de larga escala, i.e., com muitas variáveis. Essas técnicas ainda não foram testadas no Problema de Espalhamento Eletromagnético Inverso e podem ser relevantes.
		% In the literature, only traditional formulations of Evolutionary Algorithms have been considered. There are many applications of Genetic Algorithms, Differential Evolution, and Particle Swarm Optimization. However, in the literature specialized in Evolutionary Algorithms, there have been new advances and propositions which have the potential to address large-scale problems, i.e., with many variables. These techniques have not yet been tested in the Electromagnetic Inverse Scattering Problem and may be relevant.
		
		%Além de técnicas do estado-da-arte, não há pesquisas também sobre operadores evolutivos especializados para o problema. Apenas operadores genéricos são empregados. Embora sejam tenham baixíssimo custo computacional, eles não exploram caractéristicas próprias do problema. Por isso, muitas soluções com nenhum sentido físico são exploradas. Embora a garantia de sentido físico só seja possível através de simulações (custo que se deseja evitar) ou aproximações (as quais impõem certas limitações que desincentivam o uso dos algoritmos), a exploração do espaço de soluções por operadores que relacionem melhor as variáveis pode evoluir muito o desempenho destes métodos.
		% In addition to state-of-the-art techniques, there are not propositions on specialized evolutionary operators for the problem. Only generic operators have been employed. Although they have a low computational cost, they are not guided by the problem structure. Therefore, many solutions without any physical sense are explored. Although the physical meaning guarantee is only possible through simulations (a cost that we wish to avoid) or approximations (which impose certain limitations that discourage the use of these algorithms), the exploration of the space of solutions by operators that better relate the variables can improve the performance of these methods significantly. 
		
		% * Representações baseadas em geometrias definidas a priori ou em curvas de contorno possuem a desvantagem da necessidade de execução do simulador para avaliar qualquer solução do espaço de busca.
		% * Já as representações baseadas nos pixels da imagem possuem a desvantagem do alto número de variáveis.
		% * Baseadas nessas dificuldades, Salucci et al. (2022) propôs a utilização de Modelos Substitutos para diminuir o custo computacional de abordagens baseadas em curvas de contorno.
		% * Os autores proporam um Algoritmo Evolutivo Assistido por Modelos Substitutos que foi capaz de reconstruir imagens de, principalmente, espalhadores fortes.
		% * No entanto, a abordagem proposta depende por exemplo do conhecimento prévio da quantidade de espalhadores na imagem. E é necessário também não aumentar muito o número de parâmetros do modelo de contorno para não comprometer a precisão do modelo substituto.
		% * Como se trata de uma primeira abordagem de aplicação de modelos substitutos no problema, mais investigações sobre diferentes formas de aplicação são bem-vindas.
		% * Novas metodologias com um número de menor de variáveis e que fossem capaz de obter soluções iniciais melhores do que contornos aleatórios a um custo computacional podem abrir novos caminhos para a exploração do potencial da aplicação da técnica de modelos substitutos no problema.
		
		Representations based on predefined geometries or contour curves offer certain advantages but require running the forward solver to assess each solution, which can be computationally expensive. On the other hand, representations based on image pixels suffer from a high number of variables, making the optimization process more complex. To mitigate these challenges, \cite{salucci2022learned} proposed the utilization of Surrogate Models as a means to reduce the computational cost associated with contour-based approaches. They introduced a Surrogate-model Assisted Evolutionary Algorithm that demonstrated promising results in reconstructing images predominantly composed of strong scatterers, i.e., objects with high contrast or large dimensions in wavelengths. By employing surrogate models, the algorithm could approximate the behavior of the expensive simulator, enabling more efficient evaluations of solutions.
		
		However, the proposed approach does have certain dependencies and limitations. For instance, it relies on prior knowledge of the number of scatterers present in the image. Moreover, it is crucial to strike a balance in the number of parameters used in the contour model to ensure that the accuracy of the surrogate model is not compromised. Considering that \cite{salucci2022learned} was the first attempt to apply surrogate models to the problem, there is room for further investigation into alternative methods of application. Exploring different forms of utilizing surrogate models can help uncover new insights and improve the efficiency of the approach.
		
		One potential avenue for future research is the development of methodologies with a reduced number of variables. By minimizing the number of variables, the accuracy of the surrogate model becomes higher and the computational cost can be further decreased. Additionally, there is a need for techniques that can obtain superior initial solutions compared to random contours while maintaining a low computational burden. Such advancements can unlock new possibilities and allow for a more comprehensive exploration of the potential applications of surrogate models in image reconstruction.
		
		%Para poder avaliar bem o impacto que modificações têm na performance dos algoritmos, experimentações coerentes precisam ser planejadas. Embora situações arbitrárias possam ilustrar bem a capacidade dos métodos em fazer boas reconstruções e experimentos com dados reais sejam muito relevantes para atestar aplicação em situações práticas, a medição da performance dos métodos e comparações precisam seguir princípios como o controle de fatores de efeito e a aleatoridade de instâncias. Esses princípios já são bem estabelecidos na literatura especializada sobre Algoritmos Evolutionários. No entanto, esta prática é pouco difundida na literatura sobre métodos para Espalhamento Eletromagnético Inverso. Existe espaço tanto para contribuições sobre o isolamento de fatores de efeitos quanto novos indicadores que qualifiquem melhor a capacidade de reconstrução dos algoritmos. Isto se aplica não só aos métodos estocásticos como também aos determinísticos.
		In order to adequately assess the impact that modifications have on the performance of algorithms, suitable experiments need to be designed. Arbitrary situations can illustrate well the capacity of methods in making good reconstructions, i.e., reasonable shape retrievement and contrast estimation. In addition, experiments with real data are very relevant to attest the application in practical situations. However, measurement of methods performance and comparisons need to follow principles such as control of effect factors and random instances. These principles are already well established in the specialized literature on Evolutionary Algorithms. However, this practice is little widespread in the literature on methods for Electromagnetic Inverse Scattering. There are opportunities for contributions on the insulation of effects factors and new indicators that better qualify the algorithm's reconstruction capacity. This applies not only to stochastic methods but also to deterministic ones.
	
	\section{Objectives}\label{chap:introduction:objectives}
	
		%O presente texto se propõe a estabelecer e delimitar a pesquisa conduzida neste doutorado, além de oferecer uma revisão significativa da teoria e das metodologias quantitativas para o problemas (determinísticas e estocásticas). O objetivo geral desta pesquisa é investigar e propor a aplicação de técnicas que possam trazer avanços para o estado-da-arte de Algoritmos Evolutivos aplicados ao Imageamento em Microondas. A partir das investigações conduzidas neste trabalho, espera-se obter no final um algoritmo construído a partir das técnicas propostas, avaliá-lo e compará-lo com outras metodologias propostas na literatura.
		% This text proposes to establish and delimit the research conducted in this doctorate, as well as offering a significant revision of the theory and quantitative methodologies for the problem (deterministic and stochastic ones). The overall purpose of this research is to investigate and propose the application of techniques that can bring advances to the state-of-the-art Evolutionary Algorithms applied to Microwave Imaging. From the investigations conducted in this work, a final algorithm is expected. Besides its structure, suitable evaluation and comparisons against other proposed methodologies are also awaited.
		
		% * Esta tese de doutorado foi escrita com o objetivo geral de apresentar os aspectos teóricos e práticos sobre o que significa resolver o problema inverso de espalhamento eletromagnético e de contribuir na exploração de técnicas novas que podem ampliar os horizontes do Imageamento em Micro-ondas.
		% * Inicialmente, a definição das equações matemáticas que descrevem o problema são abordadas nessa tese com objetivo de reunir o conhecimento sobre os principais aspectos teóricos do problema que têm sido abordados na literatura.
		% * Além disso, é feito um esforço na revisão de diversos métodos numéricos presentes na literatura para resolver o problema. Todo caminho para a elaboração de algoritmos para o problema, que vai desde a forma de discretização até as diferentes classes e tendências dos algoritmos são contemplados.
		% * A partir da identificação das lacunas na literatura e da restrição do problema à sua forma bidimensional, dois objetivos específicos são traçados para esta pesquisa e norteiam a contribuição originalidade deste trabalho:
		% 1. Investigação de uma forma diferente das presentes no estado-da-arte para a representação de soluções a qual viabilize uma aplicação de modelos substitutos ainda mais precisa.
		% 2. Desenvolvimento de um processo de experimentação bem estruturado, capaz de avaliar melhor aspectos da reconstrução, quantificar a performance e fazer comparações robustas com outras metodologias na literatura do problema.
		
		This doctoral thesis aims to provide a comprehensive exploration of the theoretical and practical aspects involved in solving the inverse problem of electromagnetic scattering and to contribute to the advancement of Microwave Imaging techniques. The thesis systematically investigates the problem, highlighting the gaps in the existing literature and proposing novel approaches. The thesis begins by examining the foundational aspects of the problem by defining the mathematical equations that establish the electromagnetic scattering. This step allows for a thorough understanding of the theoretical aspects and establishes a solid knowledge base of the problem as documented in the existing literature.
		
		Furthermore, the thesis undertakes a comprehensive review of numerical methods that have been proposed in the literature for solving the inverse problem. It covers a range of approaches, from the discretization of the problem to the different classes and trends of algorithms employed in the field. By surveying the existing methodologies, the thesis establishes a comprehensive overview of the available techniques and lays the groundwork for the development of novel algorithms.
		
		Building on the identified gaps in the literature and the decision to focus on the two-dimensional form of the problem, the thesis outlines two specific objectives that guide the original contributions of this research.
		
		\begin{enumerate}
			\item An emphasis on investigating alternative approaches for representing solutions, aiming to enhance the application of surrogate models in a more precise and efficient manner. The proposition of a new algorithm does not presuppose any specific application, rather it is developed for a general formulation and it can be extended to several possible applications.
			\item The development of a well-structured experimentation process. This process is designed to comprehensively evaluate various aspects of image reconstruction, quantify the performance of different algorithms, and enable robust comparisons with other methodologies in the literature. 
		\end{enumerate}
		
	\section{Contributions and Novelties}\label{chap:introduction:contribution}
	
	
		% By exploring innovative ways of representing solutions, the thesis aims to contribute to the advancement of Microwave Imaging techniques and address the limitations of existing methodologies.
		
		% By establishing rigorous evaluation criteria and experimental protocols, the thesis contributes to the establishment of standardized benchmarks and facilitates meaningful comparisons between different approaches.
	
		% As principais contribuições dessa pesquisa se resumem a:
		% The main contributions of this research are summarized as follows:
		
		% \begin{itemize}
			% \item A proposta de uma metodologia de transformação do problema de inversão em um problema de otimização bidimensional a qual viabiliza uma aplicação da técnica de modelos de substitutos com mais precisão. A transformação proposta se baseia na reconstrução alcançada por métodos qualitativos e estabelece uma forma prática de atribuir contraste e refinar a geometria. Embora a adoção de métodos qualitativos para obtenção de soluções iniciais não sejam novidade, este tipo de uso nunca foi abordado na literatura. Além disso, essa abordagem se torna efetivamente viável a partir da aplicação de modelos substitutos, a qual só foi contemplada em um único trabalho na literatura até o momento. A metodologia proposta possui a vantagem de continuar alcançando boas reconstruções de espalhadores fortes como a outra aplicação de modelos substitutos proposta, só que com um número menor de variáveis a qual permite maior precisão do modelo e um tempo de execução semelhante a os dos métodos tradicionais. Desta forma, o trabalho contribui para o avanço do emprego de técnicas de modelos substitutos no problema e na aplicação em cenários de espalhadores fortes.
			% \item A proposta de uma estrutura mais robusta para o desenvolvimento, experimentação e avaliação de algoritmos para o problema. Seguindo o paradigma de Orientação-a-Objeto, um pacote foi desenvolvido para permitir um planejamento experimental sintético eficiente pois controla fatores de efeito no desempenho de algoritmos e permite a geração sistemática de conjuntos de teste que permite uma avaliação do desempenho médio de algoritmos num universo de instâncias do problema. Tal proposta é nova na literatura. Além disso, para avaliar o desempenho dos algoritmos, são reunidos diversos indicadores de qualidade além de propostos outros dois novos os quais são especializados no erro de posicionamento dos objetos e da reconstrução de suas respectivas geometrias. Por fim, processos de comparação estatística são definidos os quais permitem uma comparação mais rigorosa do desempenho médio dos algoritmos.
			%\item O desenvolvimento do estado-da-arte de Algoritmos Evolutivos aplicados ao Imageamento em Microondas a partir da aplicação de técnicas mais eficientes de busca e desenvolvimento de estratégias especializadas para a exploração de soluções. A originalidade desta contribuição se baseia no fato de que os algoritmos similares na literatura utilizam mecanismo tradicionais e operadores genéricos.
			% \item The development of a novel Evolutionary Algorithms applied to Microwave Imaging through the application of more efficient mechanisms as well as by the development of specialized strategies for the exploration of solutions. The originality of this expected contribution is based on the fact that similar algorithms in the literature use traditional mechanisms and generic operators.
			%\item Uma estrutura mais robusta para avaliação da performance de algoritmos para experimentos sintéticos, equipada com ferramentas de controle de fatores, aleatorização de instâncias e comparações estatísticas. A originalidade desta contribuição se baseia no fato de que este tipo experimentação na literatura é feita com poucas instâncias e escolhidas arbitrariamente.
			% \item A more robust structure for evaluating algorithm performance on synthetic experiments, equipped with factor control tools, instance randomization, and statistical comparisons. The originality of this expected contribution is based on the fact that traditional experimentation in the literature of Microwave Imaging are done with few and arbitrarily chosen instances.
		% \end{itemize}
	
		This research has made significant contributions in two key areas, which are summarized as follows:
		
		\begin{enumerate}
			\item \textbf{Methodology for Transforming the Inversion Problem}: \\ One of the primary contributions of this research is the proposal of a novel methodology that transforms the inversion problem into a two-dimensional optimization problem. This transformation enables the application of surrogate models with higher accuracy, improving the efficiency and effectiveness of the overall approach. The proposed methodology builds upon the reconstruction achieved by qualitative methods and provides a practical framework for assigning contrast and refining the geometry of the scatterers. While qualitative methods for obtaining initial solutions have been used before, their application in this context is unique and has not been explored in the literature. Furthermore, the proposed methodology effectively enhances the application of surrogate models, which have been minimally addressed in the literature. By employing surrogate models, the methodology achieves accurate reconstructions of strong scatterers while reducing the number of variables involved, resulting in improved model accuracy and comparable runtime to traditional methods. This contribution significantly advances the utilization of surrogate model techniques in the problem, particularly in scenarios involving strong scatterers.
			\item \textbf{Robust Structure for Algorithm Development and Evaluation:} \\ Another key contribution of this research is the proposal of a more robust framework for the development, experimentation, and evaluation of algorithms for the electromagnetic inverse scattering problem. By adopting the Object-Oriented paradigm, a software package has been developed to facilitate efficient synthetic experimental design. This package enables the control of various factors that impact algorithm performance and allows for the systematic generation of test sets to evaluate the average performance across a range of problem instances. This approach to experimental design is novel in the literature. Additionally, to assess algorithm performance, multiple quality indicators have been gathered, including the introduction of two new specialized indicators that focus on object positioning errors and the reconstruction of object geometries. Finally, the research defines statistical comparison processes that enable rigorous comparisons of algorithm performance averages. By providing a more comprehensive and standardized framework for algorithm development and evaluation, this contribution enhances the rigor and reliability of research in the field.
		\end{enumerate}
		
		In summary, this research has made significant contributions to the field of electromagnetic scattering inversion. The proposed methodology for transforming the problem into a two-dimensional optimization one, combined with the utilization of surrogate models, enables more accurate and efficient reconstructions, particularly in scenarios involving strong scatterers. Additionally, the establishment of a robust structure for algorithm development, experimentation, and evaluation enhances the reliability and comparability of research findings. These contributions enable further advancements in the use of surrogate models and the development of improved algorithms for microwave imaging.
	
	\section{Organization}\label{chap:introduction:organization}
	
		%Este texto foi escrito para apresentar o trabalho já realizado neste doutorado e defender a coerência do projeto com os requisitos do programa e com o tempo disponível. A narrativa contruída parte de uma revisão significativa da teoria e das metodologias tradicionais do problema. Baseado no levantamento dessas informações, lacunas na literatura são identificadas, perguntas são feitas, hipóteses são elaboradas e resultados preliminares são apresentados. Para isso, o texto foi estruturado da seguinte forma:
		This text aims to present an overview of the subject as well as the work already carried out in this doctorate. An effort was made on gathering and organizing relevant information concerning the mathematical theory and already proposed methodologies available in the literature. Even though some topics on the theory and some methods will not be taken into account in our proposals, their presence in the thesis is due to the purpose of providing a comprehensive discussion. Therefore, the investigation can be understood within a broader context of the literature in its current state. The text is structured as follows:
		
		\begin{itemize}
			%\item No Capítulo II, o Imageamento em Microondas é apresentado e definido em termos de um Problema de Espalhamento Eletromagnético Inverso. As equações do problema são obtidas a partir do desenvolvimento das Equações de Maxwell. Aspectos principais da teoria sobre Problemas Inversos e características principais do problema são apresentadas. Todas essas informações são levantadas a partir de uma revisão significativa da teoria do problema disponível na literatura.
			\item In Chapter \ref{chap:problemstatement}, Microwave Imaging is presented and defined in terms of an Electromagnetic Inverse Scattering Problem. The equations are obtained from the development of Maxwell's equations. The main aspects of Inverse Problems Theory and major characteristics are presented. All this information is raised from a significant review of the problem theory available in the literature.
			%\item No Capítulo III, o texto se propõe a apresentar uma revisão dos principais métodos qualitativos propostos na literatura, determinísticos e estocásticos. Primeiramente, uma padronização do problema é apresentada e, em seguida, são discutidas as formas de discretização das equações as quais são necessárias para resolver o problema computacionalmente. Além das metodologias determinísticas e estocásticas, são revistas também estratégias de linearização do problema e aplicação recentes de técnicas de Aprendizado de Máquina.
			\item In Chapter \ref{chap:methods}, the text presents a review of the numerical methods available in the literature. First, a standard domain definition is presented and then the discretization strategies are discussed since they are required to solve the problem computationally. In addition to the deterministic and stochastic quantitative methodologies, the text also presents the qualitative ones, linearization strategies, regularization methods for ill-posed systems and the recent applications of machine learning techniques.
			%\item No Capítulo IV, são identificados as principais ideias que têm sido investigadas na literatura recentemente bem como as lacunas deixadas. Baseado nessa discussão, a proposta de pesquisa é anunciada e definida em três caminhos de investigação. As perguntas e as ideias envolvendo cada uma dessas investigações são definidas. As implementações já realizadas das ideias são anunciadas e descritas. Portanto, é neste capítulo que se encontra a descrição da pesquisa proposta neste doutorado.
			\item In Chapter \ref{chap:proposed-methodology}, the most attractive ideas, which have been most investigated in the literature recently, are identified as well as the gaps left. Based on this discussion, the research proposals are announced and explained.
			%\item No Capítulo V são apresentados os resultados preliminares da aplicação das ideias implementadas. Embora as experimentações ainda não sejam o suficiente para responder adequadamente as perguntas levantadas no capítulo anterior, observações importantes sobre os resultados são feitas. Considerações, recomendações e caminhos são discutidos no fim.
			\item In Chapter \ref{chap:results}, the computational experiments are introduced, presented, and discussed. The experiments are divided into case studies and benchmarking study. Considerations and recommendations are discussed at the end.
			%\item No Capítulo VI, as considerações finais são realizadas. Neste capítulo estão presentes uma recapitulação do trabalho, uma auto-crítica e discussões sobre os próximos passos, as expectativas, as possíveis publicações resultantes da pesquisa.
			\item In Chapter \ref{chap:final}, the final considerations are carried out. This chapter is composed of the recapitulation of the work, self-criticism, discussions on the continuity proposals and the bibliographic production during the doctorate.
		\end{itemize}
		
		Chapters \ref{chap:problemstatement} and \ref{chap:methods} of the thesis provide a comprehensive overview of the theory and algorithms for microwave imaging. However, not all the concepts and methods are used in the proposed methodology in chapter \ref{chap:proposed-methodology}. The purpose of these chapters is to enhance the discussion on the subject by providing an extensive review. If the reader is interested in only the important parts for the proposed methodology, it is suggested to focus on sections \ref{chap:problemstatement:eisp}, \ref{chap:methods:definition}, \ref{chap:methods:discretization:collocation}, \ref{chap:methods:qualitative:osm}, and \ref{chap:methods:stochastic}.
		
		%Cada capítulo conta uma conclusão feita a partir do resumo e referenciamento das principais informações apresentadas. Além dos capítulos, este trabalho inclui apêndices que podem auxiliar no entendimento de aspectos teóricos do trabalho. Um resumo desses apêndices é apresentado a seguir:
		In addition to the chapters, this work also includes appendices which are auxiliary texts explaining theoretical aspects, practices, and peripheral researches. Therefore, their goal is to provide further information on some theoretical and practical aspects of this work. A summary of these appendices is presented below:
		
		\begin{itemize}
			%\item Apêndice A: um breve resumo sobre aspectos téoricos da Função Diádica de Green.
			\item Appendix \ref{app:green}: a brief review of Dyadic Green's Function.
			%\item Apêndice B: o passo-a-passo da formulação das equações integrais.
			\item Appendix \ref{app:integral}: the step-by-step formulation of the integral equations.
			%\item Apêndice C: um breve resumo sobre análise funcional.
			\item Appendix \ref{app:functional}: a brief review of functional analysis.
			%\item Anexo A: uma discussão sobre alguns problemas práticos encontrados num processo de clusterização proposto na literatura.
			%\item Appendix \ref{annex:clustering}: a discussion of some practical problems encountered in a clustering process proposed in the literature and implemented in this work.
			%\item Anexo B: uma breve ilustração e detalhes da implementação da métrica proposta para qualificar a capacidade de recuperação de forma nas imagens de contraste.
			\item Appendix \ref{annex:zetas}: a brief illustration and implementation details of the proposed metric to qualify the shape recovery capacity in contrast images.
			%\item Anexo C: descrição das relações que definem as dimensões das geometrias adotadas na implementação do processo de randomização de problemas.
			% \item Appendix \ref{annex:geometries}: description of the relationships that define the dimensions of the geometries adopted in the implementation of the instance randomization process.
			%\item Anexo D: estudo preliminar sobre a escolha adequada do valor do parâmetro de Regularização de Tikhonov. Esse estudo também foi incluído para demonstrar a aplicação da implementação da estrutura de avaliação de performance proposta em uma das investigações neste trabalho.
			% \item Appendix \ref{annex:bimreview}: a preliminary study on the proper choice of the Tikhonov regularization parameter value. This study was also included to demonstrate an application of the performance evaluation structure proposed in one of the investigations in this work.
			%\item Anexo E: estudo preliminar sobre a caracterização de um espalhador forte a partir de uma medição da relação entre tamanho e o contraste. Os resultados deste estudo baseam algumas das experimentaçõe no Capítulo V.
			% \item Appendix \ref{annex:contrastsize}: a preliminary study on the characterization of strong scatterers from measuring the relationship between size and contrast. The results of this study base some of the experiments in Chapter \ref{chap:results}.
		\end{itemize}
	
		
