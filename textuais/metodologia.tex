% ------------------------------------------------------------------------------
% Metodologia
% ------------------------------------------------------------------------------

\chapter{Metodologia}\label{chap:metodologia}

	Redigir um capítulo sobre metodologia em um trabalho acadêmico é fundamental para demonstrar a validade e a confiabilidade da pesquisa. Este capítulo deve detalhar os procedimentos e técnicas utilizados para coletar e analisar dados, permitindo que outros pesquisadores reproduzam o estudo. Aqui estão os passos essenciais para escrever um capítulo de metodologia eficaz:

	\begin{itemize}
		\item Introdução à Metodologia: Comece com uma breve introdução que esclareça o propósito do capítulo e como ele contribui para os objetivos gerais da pesquisa.
		\item Descrição da Pesquisa: Especifique o tipo de pesquisa realizada (qualitativa, quantitativa, mista) e justifique a escolha. Explique como essa abordagem é adequada para responder às perguntas de pesquisa ou hipóteses.
		\item Participantes ou Dados: Descreva a população-alvo, critérios de inclusão e exclusão, e como os participantes ou dados foram selecionados. Para pesquisas experimentais, explique como os grupos de controle e experimentais foram formados.
		\item Instrumentos e Materiais: Liste os instrumentos, ferramentas, ou materiais utilizados na coleta de dados, incluindo questionários, entrevistas, software, etc. Descreva como e por que cada instrumento foi escolhido.
		\item Procedimento: Detalhe todos os passos seguidos durante a coleta de dados. Para experimentos, descreva as condições sob as quais foram realizados, incluindo variáveis controladas e não controladas.
		\item Análise de Dados: Explique as técnicas estatísticas, métodos de análise qualitativa, ou modelos utilizados para analisar os dados coletados. Justifique a escolha desses métodos e discuta sua adequação para o tipo de dados coletados.
		\item Validade e Confiabilidade: Discuta as medidas tomadas para garantir a validade e confiabilidade dos resultados. Isso pode incluir a validação de instrumentos, triangulação de dados, ou testes piloto.
		\item Limitações: Reconheça quaisquer limitações metodológicas que possam afetar os resultados ou a interpretação da pesquisa.
		\item Ética: Se aplicável, descreva as considerações éticas relacionadas à pesquisa, incluindo aprovações de comitês de ética, consentimento informado dos participantes, e como a privacidade e a confidencialidade foram mantidas.
		\item Resumo: Conclua o capítulo com um resumo dos pontos-chave, reforçando como a metodologia adotada permite abordar as perguntas de pesquisa ou testar as hipóteses.
	\end{itemize}

	Lembre-se de que a clareza e a precisão são cruciais neste capítulo. O objetivo é fornecer informações suficientes para que outros pesquisadores possam entender como o estudo foi conduzido e, se desejado, replicar a pesquisa.