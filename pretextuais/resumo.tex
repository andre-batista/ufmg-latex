% O resumo é elemento obrigatório e consiste em um texto 
% conciso que representa os pontos relevantes do texto, devendo 
% conter de 150 a 500 palavras. Ele deve abarcar o objeto da 
% pesquisa, os objetivos, a metodologia, os resultados e a 
% conclusão.
% Abaixo do resumo, localizam-se as palavras-chaves que são 
% termos indicativos do conteúdo do trabalho e devem ser 
% precedidos da expressão Palavras-chave. São redigidas com a 
% inicial minúscula, separadas entre si com ponto-e-vírgula e finalizadas 
% com ponto final. 

\chapter*{Resumo}

	\noindent Escreva aqui o resumo do seu trabalho.

	\vspace{5mm}
	
	% Lembre-se: cada palavra-chave começando em minúscula e separadas por 
	% ponto-e-vírgula.
	\noindent\textbf{Palavras-chave}: palavra-chave 1; palavra-chave 2; palavra-chave 3.
	
	\thispagestyle{empty}