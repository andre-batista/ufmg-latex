\chapter*{Resumo}

	\noindent O Imageamento em Microondas é uma importante técnica de teste e avaliação não-destrutiva e não-invasiva com muitas aplicações em diversas áreas, como em exames médicos, triagem de segurança, sensoriamento remoto, entre outras. A técnica é baseada em um Problema Inverso de Espalhamento Eletromagnético onde as propriedades elétricas de um meio são recuperadas através de medições de campo espalhado. Além de ser um problema mal-posto, também é não-linear e multimodal. Existem vários métodos numéricos para resolver o problema e eles podem ser classificados em qualitativos ou quantitativos. Estes últimos também são classificados em métodos determinísticos ou estocásticos. Esta tese apresenta uma nova abordagem quantitativa determinística para imageamento em microondas usando algoritmos assistidos por modelos substitutos. O objetivo é abordar os desafios do problema inverso considerando a imagem qualitativa recuperada pelo Método de Amostragem de Ortogonalidade e transformando a imagem em um problema de otimização bidimensional. O método proposto se concentra em otimizar a estimativa de contraste e a operação de limiarização para minimizar o erro da equação de dados. A tese apresenta três formulações baseadas em Algoritmos Evolutivos e duas baseadas em Métodos de Direções de Busca, fornecendo um leque de opções para a resolução do problema de otimização. Além disso, uma nova estrutura é proposta para o desenvolvimento e teste de algoritmos para o problema. A estrutura inclui um pacote abrangente chamado \textit{eispy2d}, que oferece funcionalidades como geração de conjuntos de teste com controle de parâmetros, uma coleção de indicadores de desempenho (incluindo dois novos indicadores) e suporte para comparação estatística de diferentes algoritmos. Os resultados dos experimentos demonstram a eficácia dos métodos propostos. Em cenários com espalhadores fracos, os métodos propostos foram capazes de reconstruir imagens comparáveis àquelas obtidas por métodos tradicionais, enquanto alcançavam tempos de execução próximos. Além disso, em cenários mais desafiadores onde os métodos tradicionais falharam, os algoritmos propostos mostraram resultados consistentes em termos de recuperação de imagens.

	\vspace{5mm}
	
	\noindent\textbf{Palavras-chaves}: imageamento em microondas; problemas inversos; algoritmos assistidos por modelos substitutos; algoritmos evolutivos; métodos de direção de busca; biblioteca de código aberto.
	
	\thispagestyle{empty}