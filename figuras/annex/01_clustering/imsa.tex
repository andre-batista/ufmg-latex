\documentclass[11pt]{article}

    \usepackage[breakable]{tcolorbox}
    \usepackage{parskip} % Stop auto-indenting (to mimic markdown behaviour)
    
    \usepackage{iftex}
    \ifPDFTeX
    	\usepackage[T1]{fontenc}
    	\usepackage{mathpazo}
    \else
    	\usepackage{fontspec}
    \fi

    % Basic figure setup, for now with no caption control since it's done
    % automatically by Pandoc (which extracts ![](path) syntax from Markdown).
    \usepackage{graphicx}
    % Maintain compatibility with old templates. Remove in nbconvert 6.0
    \let\Oldincludegraphics\includegraphics
    % Ensure that by default, figures have no caption (until we provide a
    % proper Figure object with a Caption API and a way to capture that
    % in the conversion process - todo).
    \usepackage{caption}
    \DeclareCaptionFormat{nocaption}{}
    \captionsetup{format=nocaption,aboveskip=0pt,belowskip=0pt}

    \usepackage{float}
    \floatplacement{figure}{H} % forces figures to be placed at the correct location
    \usepackage{xcolor} % Allow colors to be defined
    \usepackage{enumerate} % Needed for markdown enumerations to work
    \usepackage{geometry} % Used to adjust the document margins
    \usepackage{amsmath} % Equations
    \usepackage{amssymb} % Equations
    \usepackage{textcomp} % defines textquotesingle
    % Hack from http://tex.stackexchange.com/a/47451/13684:
    \AtBeginDocument{%
        \def\PYZsq{\textquotesingle}% Upright quotes in Pygmentized code
    }
    \usepackage{upquote} % Upright quotes for verbatim code
    \usepackage{eurosym} % defines \euro
    \usepackage[mathletters]{ucs} % Extended unicode (utf-8) support
    \usepackage{fancyvrb} % verbatim replacement that allows latex
    \usepackage{grffile} % extends the file name processing of package graphics 
                         % to support a larger range
    \makeatletter % fix for old versions of grffile with XeLaTeX
    \@ifpackagelater{grffile}{2019/11/01}
    {
      % Do nothing on new versions
    }
    {
      \def\Gread@@xetex#1{%
        \IfFileExists{"\Gin@base".bb}%
        {\Gread@eps{\Gin@base.bb}}%
        {\Gread@@xetex@aux#1}%
      }
    }
    \makeatother
    \usepackage[Export]{adjustbox} % Used to constrain images to a maximum size
    \adjustboxset{max size={0.9\linewidth}{0.9\paperheight}}

    % The hyperref package gives us a pdf with properly built
    % internal navigation ('pdf bookmarks' for the table of contents,
    % internal cross-reference links, web links for URLs, etc.)
    \usepackage{hyperref}
    % The default LaTeX title has an obnoxious amount of whitespace. By default,
    % titling removes some of it. It also provides customization options.
    \usepackage{titling}
    \usepackage{longtable} % longtable support required by pandoc >1.10
    \usepackage{booktabs}  % table support for pandoc > 1.12.2
    \usepackage[inline]{enumitem} % IRkernel/repr support (it uses the enumerate* environment)
    \usepackage[normalem]{ulem} % ulem is needed to support strikethroughs (\sout)
                                % normalem makes italics be italics, not underlines
    \usepackage{mathrsfs}
    

    
    % Colors for the hyperref package
    \definecolor{urlcolor}{rgb}{0,.145,.698}
    \definecolor{linkcolor}{rgb}{.71,0.21,0.01}
    \definecolor{citecolor}{rgb}{.12,.54,.11}

    % ANSI colors
    \definecolor{ansi-black}{HTML}{3E424D}
    \definecolor{ansi-black-intense}{HTML}{282C36}
    \definecolor{ansi-red}{HTML}{E75C58}
    \definecolor{ansi-red-intense}{HTML}{B22B31}
    \definecolor{ansi-green}{HTML}{00A250}
    \definecolor{ansi-green-intense}{HTML}{007427}
    \definecolor{ansi-yellow}{HTML}{DDB62B}
    \definecolor{ansi-yellow-intense}{HTML}{B27D12}
    \definecolor{ansi-blue}{HTML}{208FFB}
    \definecolor{ansi-blue-intense}{HTML}{0065CA}
    \definecolor{ansi-magenta}{HTML}{D160C4}
    \definecolor{ansi-magenta-intense}{HTML}{A03196}
    \definecolor{ansi-cyan}{HTML}{60C6C8}
    \definecolor{ansi-cyan-intense}{HTML}{258F8F}
    \definecolor{ansi-white}{HTML}{C5C1B4}
    \definecolor{ansi-white-intense}{HTML}{A1A6B2}
    \definecolor{ansi-default-inverse-fg}{HTML}{FFFFFF}
    \definecolor{ansi-default-inverse-bg}{HTML}{000000}

    % common color for the border for error outputs.
    \definecolor{outerrorbackground}{HTML}{FFDFDF}

    % commands and environments needed by pandoc snippets
    % extracted from the output of `pandoc -s`
    \providecommand{\tightlist}{%
      \setlength{\itemsep}{0pt}\setlength{\parskip}{0pt}}
    \DefineVerbatimEnvironment{Highlighting}{Verbatim}{commandchars=\\\{\}}
    % Add ',fontsize=\small' for more characters per line
    \newenvironment{Shaded}{}{}
    \newcommand{\KeywordTok}[1]{\textcolor[rgb]{0.00,0.44,0.13}{\textbf{{#1}}}}
    \newcommand{\DataTypeTok}[1]{\textcolor[rgb]{0.56,0.13,0.00}{{#1}}}
    \newcommand{\DecValTok}[1]{\textcolor[rgb]{0.25,0.63,0.44}{{#1}}}
    \newcommand{\BaseNTok}[1]{\textcolor[rgb]{0.25,0.63,0.44}{{#1}}}
    \newcommand{\FloatTok}[1]{\textcolor[rgb]{0.25,0.63,0.44}{{#1}}}
    \newcommand{\CharTok}[1]{\textcolor[rgb]{0.25,0.44,0.63}{{#1}}}
    \newcommand{\StringTok}[1]{\textcolor[rgb]{0.25,0.44,0.63}{{#1}}}
    \newcommand{\CommentTok}[1]{\textcolor[rgb]{0.38,0.63,0.69}{\textit{{#1}}}}
    \newcommand{\OtherTok}[1]{\textcolor[rgb]{0.00,0.44,0.13}{{#1}}}
    \newcommand{\AlertTok}[1]{\textcolor[rgb]{1.00,0.00,0.00}{\textbf{{#1}}}}
    \newcommand{\FunctionTok}[1]{\textcolor[rgb]{0.02,0.16,0.49}{{#1}}}
    \newcommand{\RegionMarkerTok}[1]{{#1}}
    \newcommand{\ErrorTok}[1]{\textcolor[rgb]{1.00,0.00,0.00}{\textbf{{#1}}}}
    \newcommand{\NormalTok}[1]{{#1}}
    
    % Additional commands for more recent versions of Pandoc
    \newcommand{\ConstantTok}[1]{\textcolor[rgb]{0.53,0.00,0.00}{{#1}}}
    \newcommand{\SpecialCharTok}[1]{\textcolor[rgb]{0.25,0.44,0.63}{{#1}}}
    \newcommand{\VerbatimStringTok}[1]{\textcolor[rgb]{0.25,0.44,0.63}{{#1}}}
    \newcommand{\SpecialStringTok}[1]{\textcolor[rgb]{0.73,0.40,0.53}{{#1}}}
    \newcommand{\ImportTok}[1]{{#1}}
    \newcommand{\DocumentationTok}[1]{\textcolor[rgb]{0.73,0.13,0.13}{\textit{{#1}}}}
    \newcommand{\AnnotationTok}[1]{\textcolor[rgb]{0.38,0.63,0.69}{\textbf{\textit{{#1}}}}}
    \newcommand{\CommentVarTok}[1]{\textcolor[rgb]{0.38,0.63,0.69}{\textbf{\textit{{#1}}}}}
    \newcommand{\VariableTok}[1]{\textcolor[rgb]{0.10,0.09,0.49}{{#1}}}
    \newcommand{\ControlFlowTok}[1]{\textcolor[rgb]{0.00,0.44,0.13}{\textbf{{#1}}}}
    \newcommand{\OperatorTok}[1]{\textcolor[rgb]{0.40,0.40,0.40}{{#1}}}
    \newcommand{\BuiltInTok}[1]{{#1}}
    \newcommand{\ExtensionTok}[1]{{#1}}
    \newcommand{\PreprocessorTok}[1]{\textcolor[rgb]{0.74,0.48,0.00}{{#1}}}
    \newcommand{\AttributeTok}[1]{\textcolor[rgb]{0.49,0.56,0.16}{{#1}}}
    \newcommand{\InformationTok}[1]{\textcolor[rgb]{0.38,0.63,0.69}{\textbf{\textit{{#1}}}}}
    \newcommand{\WarningTok}[1]{\textcolor[rgb]{0.38,0.63,0.69}{\textbf{\textit{{#1}}}}}
    
    
    % Define a nice break command that doesn't care if a line doesn't already
    % exist.
    \def\br{\hspace*{\fill} \\* }
    % Math Jax compatibility definitions
    \def\gt{>}
    \def\lt{<}
    \let\Oldtex\TeX
    \let\Oldlatex\LaTeX
    \renewcommand{\TeX}{\textrm{\Oldtex}}
    \renewcommand{\LaTeX}{\textrm{\Oldlatex}}
    % Document parameters
    % Document title
    \title{imsa}
    
    
    
    
    
% Pygments definitions
\makeatletter
\def\PY@reset{\let\PY@it=\relax \let\PY@bf=\relax%
    \let\PY@ul=\relax \let\PY@tc=\relax%
    \let\PY@bc=\relax \let\PY@ff=\relax}
\def\PY@tok#1{\csname PY@tok@#1\endcsname}
\def\PY@toks#1+{\ifx\relax#1\empty\else%
    \PY@tok{#1}\expandafter\PY@toks\fi}
\def\PY@do#1{\PY@bc{\PY@tc{\PY@ul{%
    \PY@it{\PY@bf{\PY@ff{#1}}}}}}}
\def\PY#1#2{\PY@reset\PY@toks#1+\relax+\PY@do{#2}}

\@namedef{PY@tok@w}{\def\PY@tc##1{\textcolor[rgb]{0.73,0.73,0.73}{##1}}}
\@namedef{PY@tok@c}{\let\PY@it=\textit\def\PY@tc##1{\textcolor[rgb]{0.25,0.50,0.50}{##1}}}
\@namedef{PY@tok@cp}{\def\PY@tc##1{\textcolor[rgb]{0.74,0.48,0.00}{##1}}}
\@namedef{PY@tok@k}{\let\PY@bf=\textbf\def\PY@tc##1{\textcolor[rgb]{0.00,0.50,0.00}{##1}}}
\@namedef{PY@tok@kp}{\def\PY@tc##1{\textcolor[rgb]{0.00,0.50,0.00}{##1}}}
\@namedef{PY@tok@kt}{\def\PY@tc##1{\textcolor[rgb]{0.69,0.00,0.25}{##1}}}
\@namedef{PY@tok@o}{\def\PY@tc##1{\textcolor[rgb]{0.40,0.40,0.40}{##1}}}
\@namedef{PY@tok@ow}{\let\PY@bf=\textbf\def\PY@tc##1{\textcolor[rgb]{0.67,0.13,1.00}{##1}}}
\@namedef{PY@tok@nb}{\def\PY@tc##1{\textcolor[rgb]{0.00,0.50,0.00}{##1}}}
\@namedef{PY@tok@nf}{\def\PY@tc##1{\textcolor[rgb]{0.00,0.00,1.00}{##1}}}
\@namedef{PY@tok@nc}{\let\PY@bf=\textbf\def\PY@tc##1{\textcolor[rgb]{0.00,0.00,1.00}{##1}}}
\@namedef{PY@tok@nn}{\let\PY@bf=\textbf\def\PY@tc##1{\textcolor[rgb]{0.00,0.00,1.00}{##1}}}
\@namedef{PY@tok@ne}{\let\PY@bf=\textbf\def\PY@tc##1{\textcolor[rgb]{0.82,0.25,0.23}{##1}}}
\@namedef{PY@tok@nv}{\def\PY@tc##1{\textcolor[rgb]{0.10,0.09,0.49}{##1}}}
\@namedef{PY@tok@no}{\def\PY@tc##1{\textcolor[rgb]{0.53,0.00,0.00}{##1}}}
\@namedef{PY@tok@nl}{\def\PY@tc##1{\textcolor[rgb]{0.63,0.63,0.00}{##1}}}
\@namedef{PY@tok@ni}{\let\PY@bf=\textbf\def\PY@tc##1{\textcolor[rgb]{0.60,0.60,0.60}{##1}}}
\@namedef{PY@tok@na}{\def\PY@tc##1{\textcolor[rgb]{0.49,0.56,0.16}{##1}}}
\@namedef{PY@tok@nt}{\let\PY@bf=\textbf\def\PY@tc##1{\textcolor[rgb]{0.00,0.50,0.00}{##1}}}
\@namedef{PY@tok@nd}{\def\PY@tc##1{\textcolor[rgb]{0.67,0.13,1.00}{##1}}}
\@namedef{PY@tok@s}{\def\PY@tc##1{\textcolor[rgb]{0.73,0.13,0.13}{##1}}}
\@namedef{PY@tok@sd}{\let\PY@it=\textit\def\PY@tc##1{\textcolor[rgb]{0.73,0.13,0.13}{##1}}}
\@namedef{PY@tok@si}{\let\PY@bf=\textbf\def\PY@tc##1{\textcolor[rgb]{0.73,0.40,0.53}{##1}}}
\@namedef{PY@tok@se}{\let\PY@bf=\textbf\def\PY@tc##1{\textcolor[rgb]{0.73,0.40,0.13}{##1}}}
\@namedef{PY@tok@sr}{\def\PY@tc##1{\textcolor[rgb]{0.73,0.40,0.53}{##1}}}
\@namedef{PY@tok@ss}{\def\PY@tc##1{\textcolor[rgb]{0.10,0.09,0.49}{##1}}}
\@namedef{PY@tok@sx}{\def\PY@tc##1{\textcolor[rgb]{0.00,0.50,0.00}{##1}}}
\@namedef{PY@tok@m}{\def\PY@tc##1{\textcolor[rgb]{0.40,0.40,0.40}{##1}}}
\@namedef{PY@tok@gh}{\let\PY@bf=\textbf\def\PY@tc##1{\textcolor[rgb]{0.00,0.00,0.50}{##1}}}
\@namedef{PY@tok@gu}{\let\PY@bf=\textbf\def\PY@tc##1{\textcolor[rgb]{0.50,0.00,0.50}{##1}}}
\@namedef{PY@tok@gd}{\def\PY@tc##1{\textcolor[rgb]{0.63,0.00,0.00}{##1}}}
\@namedef{PY@tok@gi}{\def\PY@tc##1{\textcolor[rgb]{0.00,0.63,0.00}{##1}}}
\@namedef{PY@tok@gr}{\def\PY@tc##1{\textcolor[rgb]{1.00,0.00,0.00}{##1}}}
\@namedef{PY@tok@ge}{\let\PY@it=\textit}
\@namedef{PY@tok@gs}{\let\PY@bf=\textbf}
\@namedef{PY@tok@gp}{\let\PY@bf=\textbf\def\PY@tc##1{\textcolor[rgb]{0.00,0.00,0.50}{##1}}}
\@namedef{PY@tok@go}{\def\PY@tc##1{\textcolor[rgb]{0.53,0.53,0.53}{##1}}}
\@namedef{PY@tok@gt}{\def\PY@tc##1{\textcolor[rgb]{0.00,0.27,0.87}{##1}}}
\@namedef{PY@tok@err}{\def\PY@bc##1{{\setlength{\fboxsep}{\string -\fboxrule}\fcolorbox[rgb]{1.00,0.00,0.00}{1,1,1}{\strut ##1}}}}
\@namedef{PY@tok@kc}{\let\PY@bf=\textbf\def\PY@tc##1{\textcolor[rgb]{0.00,0.50,0.00}{##1}}}
\@namedef{PY@tok@kd}{\let\PY@bf=\textbf\def\PY@tc##1{\textcolor[rgb]{0.00,0.50,0.00}{##1}}}
\@namedef{PY@tok@kn}{\let\PY@bf=\textbf\def\PY@tc##1{\textcolor[rgb]{0.00,0.50,0.00}{##1}}}
\@namedef{PY@tok@kr}{\let\PY@bf=\textbf\def\PY@tc##1{\textcolor[rgb]{0.00,0.50,0.00}{##1}}}
\@namedef{PY@tok@bp}{\def\PY@tc##1{\textcolor[rgb]{0.00,0.50,0.00}{##1}}}
\@namedef{PY@tok@fm}{\def\PY@tc##1{\textcolor[rgb]{0.00,0.00,1.00}{##1}}}
\@namedef{PY@tok@vc}{\def\PY@tc##1{\textcolor[rgb]{0.10,0.09,0.49}{##1}}}
\@namedef{PY@tok@vg}{\def\PY@tc##1{\textcolor[rgb]{0.10,0.09,0.49}{##1}}}
\@namedef{PY@tok@vi}{\def\PY@tc##1{\textcolor[rgb]{0.10,0.09,0.49}{##1}}}
\@namedef{PY@tok@vm}{\def\PY@tc##1{\textcolor[rgb]{0.10,0.09,0.49}{##1}}}
\@namedef{PY@tok@sa}{\def\PY@tc##1{\textcolor[rgb]{0.73,0.13,0.13}{##1}}}
\@namedef{PY@tok@sb}{\def\PY@tc##1{\textcolor[rgb]{0.73,0.13,0.13}{##1}}}
\@namedef{PY@tok@sc}{\def\PY@tc##1{\textcolor[rgb]{0.73,0.13,0.13}{##1}}}
\@namedef{PY@tok@dl}{\def\PY@tc##1{\textcolor[rgb]{0.73,0.13,0.13}{##1}}}
\@namedef{PY@tok@s2}{\def\PY@tc##1{\textcolor[rgb]{0.73,0.13,0.13}{##1}}}
\@namedef{PY@tok@sh}{\def\PY@tc##1{\textcolor[rgb]{0.73,0.13,0.13}{##1}}}
\@namedef{PY@tok@s1}{\def\PY@tc##1{\textcolor[rgb]{0.73,0.13,0.13}{##1}}}
\@namedef{PY@tok@mb}{\def\PY@tc##1{\textcolor[rgb]{0.40,0.40,0.40}{##1}}}
\@namedef{PY@tok@mf}{\def\PY@tc##1{\textcolor[rgb]{0.40,0.40,0.40}{##1}}}
\@namedef{PY@tok@mh}{\def\PY@tc##1{\textcolor[rgb]{0.40,0.40,0.40}{##1}}}
\@namedef{PY@tok@mi}{\def\PY@tc##1{\textcolor[rgb]{0.40,0.40,0.40}{##1}}}
\@namedef{PY@tok@il}{\def\PY@tc##1{\textcolor[rgb]{0.40,0.40,0.40}{##1}}}
\@namedef{PY@tok@mo}{\def\PY@tc##1{\textcolor[rgb]{0.40,0.40,0.40}{##1}}}
\@namedef{PY@tok@ch}{\let\PY@it=\textit\def\PY@tc##1{\textcolor[rgb]{0.25,0.50,0.50}{##1}}}
\@namedef{PY@tok@cm}{\let\PY@it=\textit\def\PY@tc##1{\textcolor[rgb]{0.25,0.50,0.50}{##1}}}
\@namedef{PY@tok@cpf}{\let\PY@it=\textit\def\PY@tc##1{\textcolor[rgb]{0.25,0.50,0.50}{##1}}}
\@namedef{PY@tok@c1}{\let\PY@it=\textit\def\PY@tc##1{\textcolor[rgb]{0.25,0.50,0.50}{##1}}}
\@namedef{PY@tok@cs}{\let\PY@it=\textit\def\PY@tc##1{\textcolor[rgb]{0.25,0.50,0.50}{##1}}}

\def\PYZbs{\char`\\}
\def\PYZus{\char`\_}
\def\PYZob{\char`\{}
\def\PYZcb{\char`\}}
\def\PYZca{\char`\^}
\def\PYZam{\char`\&}
\def\PYZlt{\char`\<}
\def\PYZgt{\char`\>}
\def\PYZsh{\char`\#}
\def\PYZpc{\char`\%}
\def\PYZdl{\char`\$}
\def\PYZhy{\char`\-}
\def\PYZsq{\char`\'}
\def\PYZdq{\char`\"}
\def\PYZti{\char`\~}
% for compatibility with earlier versions
\def\PYZat{@}
\def\PYZlb{[}
\def\PYZrb{]}
\makeatother


    % For linebreaks inside Verbatim environment from package fancyvrb. 
    \makeatletter
        \newbox\Wrappedcontinuationbox 
        \newbox\Wrappedvisiblespacebox 
        \newcommand*\Wrappedvisiblespace {\textcolor{red}{\textvisiblespace}} 
        \newcommand*\Wrappedcontinuationsymbol {\textcolor{red}{\llap{\tiny$\m@th\hookrightarrow$}}} 
        \newcommand*\Wrappedcontinuationindent {3ex } 
        \newcommand*\Wrappedafterbreak {\kern\Wrappedcontinuationindent\copy\Wrappedcontinuationbox} 
        % Take advantage of the already applied Pygments mark-up to insert 
        % potential linebreaks for TeX processing. 
        %        {, <, #, %, $, ' and ": go to next line. 
        %        _, }, ^, &, >, - and ~: stay at end of broken line. 
        % Use of \textquotesingle for straight quote. 
        \newcommand*\Wrappedbreaksatspecials {% 
            \def\PYGZus{\discretionary{\char`\_}{\Wrappedafterbreak}{\char`\_}}% 
            \def\PYGZob{\discretionary{}{\Wrappedafterbreak\char`\{}{\char`\{}}% 
            \def\PYGZcb{\discretionary{\char`\}}{\Wrappedafterbreak}{\char`\}}}% 
            \def\PYGZca{\discretionary{\char`\^}{\Wrappedafterbreak}{\char`\^}}% 
            \def\PYGZam{\discretionary{\char`\&}{\Wrappedafterbreak}{\char`\&}}% 
            \def\PYGZlt{\discretionary{}{\Wrappedafterbreak\char`\<}{\char`\<}}% 
            \def\PYGZgt{\discretionary{\char`\>}{\Wrappedafterbreak}{\char`\>}}% 
            \def\PYGZsh{\discretionary{}{\Wrappedafterbreak\char`\#}{\char`\#}}% 
            \def\PYGZpc{\discretionary{}{\Wrappedafterbreak\char`\%}{\char`\%}}% 
            \def\PYGZdl{\discretionary{}{\Wrappedafterbreak\char`\$}{\char`\$}}% 
            \def\PYGZhy{\discretionary{\char`\-}{\Wrappedafterbreak}{\char`\-}}% 
            \def\PYGZsq{\discretionary{}{\Wrappedafterbreak\textquotesingle}{\textquotesingle}}% 
            \def\PYGZdq{\discretionary{}{\Wrappedafterbreak\char`\"}{\char`\"}}% 
            \def\PYGZti{\discretionary{\char`\~}{\Wrappedafterbreak}{\char`\~}}% 
        } 
        % Some characters . , ; ? ! / are not pygmentized. 
        % This macro makes them "active" and they will insert potential linebreaks 
        \newcommand*\Wrappedbreaksatpunct {% 
            \lccode`\~`\.\lowercase{\def~}{\discretionary{\hbox{\char`\.}}{\Wrappedafterbreak}{\hbox{\char`\.}}}% 
            \lccode`\~`\,\lowercase{\def~}{\discretionary{\hbox{\char`\,}}{\Wrappedafterbreak}{\hbox{\char`\,}}}% 
            \lccode`\~`\;\lowercase{\def~}{\discretionary{\hbox{\char`\;}}{\Wrappedafterbreak}{\hbox{\char`\;}}}% 
            \lccode`\~`\:\lowercase{\def~}{\discretionary{\hbox{\char`\:}}{\Wrappedafterbreak}{\hbox{\char`\:}}}% 
            \lccode`\~`\?\lowercase{\def~}{\discretionary{\hbox{\char`\?}}{\Wrappedafterbreak}{\hbox{\char`\?}}}% 
            \lccode`\~`\!\lowercase{\def~}{\discretionary{\hbox{\char`\!}}{\Wrappedafterbreak}{\hbox{\char`\!}}}% 
            \lccode`\~`\/\lowercase{\def~}{\discretionary{\hbox{\char`\/}}{\Wrappedafterbreak}{\hbox{\char`\/}}}% 
            \catcode`\.\active
            \catcode`\,\active 
            \catcode`\;\active
            \catcode`\:\active
            \catcode`\?\active
            \catcode`\!\active
            \catcode`\/\active 
            \lccode`\~`\~ 	
        }
    \makeatother

    \let\OriginalVerbatim=\Verbatim
    \makeatletter
    \renewcommand{\Verbatim}[1][1]{%
        %\parskip\z@skip
        \sbox\Wrappedcontinuationbox {\Wrappedcontinuationsymbol}%
        \sbox\Wrappedvisiblespacebox {\FV@SetupFont\Wrappedvisiblespace}%
        \def\FancyVerbFormatLine ##1{\hsize\linewidth
            \vtop{\raggedright\hyphenpenalty\z@\exhyphenpenalty\z@
                \doublehyphendemerits\z@\finalhyphendemerits\z@
                \strut ##1\strut}%
        }%
        % If the linebreak is at a space, the latter will be displayed as visible
        % space at end of first line, and a continuation symbol starts next line.
        % Stretch/shrink are however usually zero for typewriter font.
        \def\FV@Space {%
            \nobreak\hskip\z@ plus\fontdimen3\font minus\fontdimen4\font
            \discretionary{\copy\Wrappedvisiblespacebox}{\Wrappedafterbreak}
            {\kern\fontdimen2\font}%
        }%
        
        % Allow breaks at special characters using \PYG... macros.
        \Wrappedbreaksatspecials
        % Breaks at punctuation characters . , ; ? ! and / need catcode=\active 	
        \OriginalVerbatim[#1,codes*=\Wrappedbreaksatpunct]%
    }
    \makeatother

    % Exact colors from NB
    \definecolor{incolor}{HTML}{303F9F}
    \definecolor{outcolor}{HTML}{D84315}
    \definecolor{cellborder}{HTML}{CFCFCF}
    \definecolor{cellbackground}{HTML}{F7F7F7}
    
    % prompt
    \makeatletter
    \newcommand{\boxspacing}{\kern\kvtcb@left@rule\kern\kvtcb@boxsep}
    \makeatother
    \newcommand{\prompt}[4]{
        {\ttfamily\llap{{\color{#2}[#3]:\hspace{3pt}#4}}\vspace{-\baselineskip}}
    }
    

    
    % Prevent overflowing lines due to hard-to-break entities
    \sloppy 
    % Setup hyperref package
    \hypersetup{
      breaklinks=true,  % so long urls are correctly broken across lines
      colorlinks=true,
      urlcolor=urlcolor,
      linkcolor=linkcolor,
      citecolor=citecolor,
      }
    % Slightly bigger margins than the latex defaults
    
    \geometry{verbose,tmargin=1in,bmargin=1in,lmargin=1in,rmargin=1in}
    
    

\begin{document}
    
    \maketitle
    
    

    
    \begin{tcolorbox}[breakable, size=fbox, boxrule=1pt, pad at break*=1mm,colback=cellbackground, colframe=cellborder]
\prompt{In}{incolor}{1}{\boxspacing}
\begin{Verbatim}[commandchars=\\\{\}]
\PY{k+kn}{import} \PY{n+nn}{numpy} \PY{k}{as} \PY{n+nn}{np}
\PY{k+kn}{from} \PY{n+nn}{matplotlib} \PY{k+kn}{import} \PY{n}{pyplot} \PY{k}{as} \PY{n}{plt}
\end{Verbatim}
\end{tcolorbox}

    \hypertarget{clustering-procedure}{%
\section{Clustering Procedure}\label{clustering-procedure}}

This notebooks is an attempt to implement the clustering procedure
defined by \emph{Caorsi et a. (2004)}. The example illustrated in the
paper is addressed in order to replicate what was obtained in the
figures.

Firstly, it will be define a routine to return indexes of the neighbors
cells for a given one. This is a quick shortcut in order to make the
code cleaner. The neighborhood structure may be define as 4 cells (left,
right, top, and bottom) and 8 cells (northwest, northeast, soultheast,
soulthwest).

    \begin{tcolorbox}[breakable, size=fbox, boxrule=1pt, pad at break*=1mm,colback=cellbackground, colframe=cellborder]
\prompt{In}{incolor}{2}{\boxspacing}
\begin{Verbatim}[commandchars=\\\{\}]
\PY{k}{def} \PY{n+nf}{neighbors}\PY{p}{(}\PY{n}{i}\PY{p}{,} \PY{n}{j}\PY{p}{,} \PY{n}{I}\PY{p}{,} \PY{n}{J}\PY{p}{,} \PY{n}{neighbors}\PY{o}{=}\PY{l+m+mi}{8}\PY{p}{)}\PY{p}{:}
    
    \PY{k}{if} \PY{n}{neighbors} \PY{o}{==} \PY{l+m+mi}{4}\PY{p}{:}
        \PY{n}{possible\PYZus{}i} \PY{o}{=} \PY{p}{[}\PY{n}{i}\PY{o}{\PYZhy{}}\PY{l+m+mi}{1}\PY{p}{,} \PY{n}{i}\PY{o}{+}\PY{l+m+mi}{1}\PY{p}{,}   \PY{n}{i}\PY{p}{,}   \PY{n}{i}\PY{p}{]}
        \PY{n}{possible\PYZus{}j} \PY{o}{=} \PY{p}{[}\PY{n}{j}\PY{p}{,}     \PY{n}{j}\PY{p}{,} \PY{n}{j}\PY{o}{\PYZhy{}}\PY{l+m+mi}{1}\PY{p}{,} \PY{n}{j}\PY{o}{+}\PY{l+m+mi}{1}\PY{p}{]}
    \PY{k}{elif} \PY{n}{neighbors} \PY{o}{==} \PY{l+m+mi}{8}\PY{p}{:}
        \PY{n}{possible\PYZus{}i} \PY{o}{=} \PY{p}{[}\PY{n}{i}\PY{o}{\PYZhy{}}\PY{l+m+mi}{1}\PY{p}{,} \PY{n}{i}\PY{o}{\PYZhy{}}\PY{l+m+mi}{1}\PY{p}{,} \PY{n}{i}\PY{o}{\PYZhy{}}\PY{l+m+mi}{1}\PY{p}{,}   \PY{n}{i}\PY{p}{,}   \PY{n}{i}\PY{p}{,} \PY{n}{i}\PY{o}{+}\PY{l+m+mi}{1}\PY{p}{,} \PY{n}{i}\PY{o}{+}\PY{l+m+mi}{1}\PY{p}{,} \PY{n}{i}\PY{o}{+}\PY{l+m+mi}{1}\PY{p}{]}
        \PY{n}{possible\PYZus{}j} \PY{o}{=} \PY{p}{[}\PY{n}{j}\PY{o}{\PYZhy{}}\PY{l+m+mi}{1}\PY{p}{,}   \PY{n}{j}\PY{p}{,} \PY{n}{j}\PY{o}{+}\PY{l+m+mi}{1}\PY{p}{,} \PY{n}{j}\PY{o}{\PYZhy{}}\PY{l+m+mi}{1}\PY{p}{,} \PY{n}{j}\PY{o}{+}\PY{l+m+mi}{1}\PY{p}{,} \PY{n}{j}\PY{o}{\PYZhy{}}\PY{l+m+mi}{1}\PY{p}{,}   \PY{n}{j}\PY{p}{,} \PY{n}{j}\PY{o}{+}\PY{l+m+mi}{1}\PY{p}{]}

    \PY{n}{ilist}\PY{p}{,} \PY{n}{jlist} \PY{o}{=} \PY{p}{[}\PY{p}{]}\PY{p}{,} \PY{p}{[}\PY{p}{]}
    \PY{k}{for} \PY{n}{n} \PY{o+ow}{in} \PY{n+nb}{range}\PY{p}{(}\PY{n+nb}{len}\PY{p}{(}\PY{n}{possible\PYZus{}i}\PY{p}{)}\PY{p}{)}\PY{p}{:}
        \PY{k}{if} \PY{n}{possible\PYZus{}i}\PY{p}{[}\PY{n}{n}\PY{p}{]} \PY{o}{\PYZlt{}} \PY{l+m+mi}{0} \PY{o+ow}{or} \PY{n}{possible\PYZus{}i}\PY{p}{[}\PY{n}{n}\PY{p}{]} \PY{o}{==} \PY{n}{I} \PY{o+ow}{or} \PY{n}{possible\PYZus{}j}\PY{p}{[}\PY{n}{n}\PY{p}{]} \PY{o}{\PYZlt{}} \PY{l+m+mi}{0} \PY{o+ow}{or} \PY{n}{possible\PYZus{}j}\PY{p}{[}\PY{n}{n}\PY{p}{]} \PY{o}{==} \PY{n}{J}\PY{p}{:}
            \PY{k}{pass}
        \PY{k}{else}\PY{p}{:}
            \PY{n}{ilist}\PY{o}{.}\PY{n}{append}\PY{p}{(}\PY{n}{possible\PYZus{}i}\PY{p}{[}\PY{n}{n}\PY{p}{]}\PY{p}{)}
            \PY{n}{jlist}\PY{o}{.}\PY{n}{append}\PY{p}{(}\PY{n}{possible\PYZus{}j}\PY{p}{[}\PY{n}{n}\PY{p}{]}\PY{p}{)}
    \PY{k}{return} \PY{n}{ilist}\PY{p}{,} \PY{n}{jlist}
\end{Verbatim}
\end{tcolorbox}

    \hypertarget{image-definition}{%
\subsection{Image Definition}\label{image-definition}}

The Figures 2(a) and 2(b) are replicated in the next code. Actually,
there is a slight different in some values, as one may note in the
histogram. We have tried to obtain the same histogram curve, but it was
not accomplished. We decided to define the image in order to obtain the
same minimum value, which will be used as threshold level. Therefore, we
define it to at least obtain the same image after thresholding.

    \begin{tcolorbox}[breakable, size=fbox, boxrule=1pt, pad at break*=1mm,colback=cellbackground, colframe=cellborder]
\prompt{In}{incolor}{3}{\boxspacing}
\begin{Verbatim}[commandchars=\\\{\}]
\PY{c+c1}{\PYZsh{} Image array}
\PY{n}{image} \PY{o}{=} \PY{n}{np}\PY{o}{.}\PY{n}{array}\PY{p}{(}\PY{p}{[}\PY{p}{[} \PY{l+m+mi}{0}\PY{p}{,}    \PY{l+m+mi}{0}\PY{p}{,}    \PY{l+m+mi}{0}\PY{p}{,}    \PY{l+m+mi}{0}\PY{p}{,}    \PY{l+m+mi}{0}\PY{p}{,}    \PY{l+m+mi}{0}\PY{p}{,}    \PY{l+m+mi}{0}\PY{p}{,}    \PY{l+m+mi}{0}\PY{p}{,}    \PY{l+m+mi}{0}\PY{p}{,}    \PY{l+m+mi}{0}\PY{p}{,} \PY{l+m+mi}{0}\PY{p}{]}\PY{p}{,}
                  \PY{p}{[} \PY{l+m+mi}{0}\PY{p}{,}  \PY{o}{.}\PY{l+m+mi}{02}\PY{p}{,}    \PY{l+m+mi}{0}\PY{p}{,}    \PY{l+m+mi}{0}\PY{p}{,}    \PY{l+m+mi}{0}\PY{p}{,}    \PY{l+m+mi}{0}\PY{p}{,}    \PY{l+m+mi}{0}\PY{p}{,}  \PY{l+m+mf}{0.2}\PY{p}{,}    \PY{l+m+mi}{0}\PY{p}{,}    \PY{l+m+mi}{0}\PY{p}{,} \PY{l+m+mi}{0}\PY{p}{]}\PY{p}{,}
                  \PY{p}{[} \PY{l+m+mi}{0}\PY{p}{,}    \PY{l+m+mi}{0}\PY{p}{,}  \PY{o}{.}\PY{l+m+mi}{04}\PY{p}{,}    \PY{l+m+mi}{0}\PY{p}{,}    \PY{l+m+mi}{0}\PY{p}{,}  \PY{o}{.}\PY{l+m+mi}{05}\PY{p}{,}  \PY{o}{.}\PY{l+m+mi}{45}\PY{p}{,}  \PY{o}{.}\PY{l+m+mi}{56}\PY{p}{,}    \PY{l+m+mi}{0}\PY{p}{,}    \PY{l+m+mi}{0}\PY{p}{,} \PY{l+m+mi}{0}\PY{p}{]}\PY{p}{,}
                  \PY{p}{[} \PY{l+m+mi}{0}\PY{p}{,}    \PY{l+m+mi}{0}\PY{p}{,}    \PY{l+m+mi}{0}\PY{p}{,}    \PY{l+m+mi}{0}\PY{p}{,}    \PY{l+m+mi}{0}\PY{p}{,}   \PY{o}{.}\PY{l+m+mi}{1}\PY{p}{,}  \PY{o}{.}\PY{l+m+mi}{56}\PY{p}{,}  \PY{o}{.}\PY{l+m+mi}{16}\PY{p}{,}  \PY{o}{.}\PY{l+m+mi}{55}\PY{p}{,}  \PY{l+m+mf}{0.2}\PY{p}{,} \PY{l+m+mi}{0}\PY{p}{]}\PY{p}{,}
                  \PY{p}{[} \PY{l+m+mi}{0}\PY{p}{,}    \PY{l+m+mi}{0}\PY{p}{,}    \PY{l+m+mi}{0}\PY{p}{,}    \PY{l+m+mi}{0}\PY{p}{,}  \PY{o}{.}\PY{l+m+mi}{08}\PY{p}{,}  \PY{o}{.}\PY{l+m+mi}{05}\PY{p}{,}  \PY{o}{.}\PY{l+m+mi}{56}\PY{p}{,}  \PY{o}{.}\PY{l+m+mi}{56}\PY{p}{,}  \PY{o}{.}\PY{l+m+mi}{45}\PY{p}{,}    \PY{l+m+mi}{0}\PY{p}{,} \PY{l+m+mi}{0}\PY{p}{]}\PY{p}{,}
                  \PY{p}{[} \PY{l+m+mi}{0}\PY{p}{,}    \PY{l+m+mi}{0}\PY{p}{,}    \PY{l+m+mi}{0}\PY{p}{,}  \PY{l+m+mf}{0.1}\PY{p}{,}  \PY{o}{.}\PY{l+m+mi}{02}\PY{p}{,}  \PY{o}{.}\PY{l+m+mi}{08}\PY{p}{,}  \PY{o}{.}\PY{l+m+mi}{02}\PY{p}{,}  \PY{o}{.}\PY{l+m+mi}{08}\PY{p}{,}    \PY{l+m+mi}{0}\PY{p}{,}    \PY{l+m+mi}{0}\PY{p}{,} \PY{l+m+mi}{0}\PY{p}{]}\PY{p}{,}
                  \PY{p}{[} \PY{l+m+mi}{0}\PY{p}{,}    \PY{l+m+mi}{0}\PY{p}{,}  \PY{o}{.}\PY{l+m+mi}{45}\PY{p}{,}  \PY{o}{.}\PY{l+m+mi}{58}\PY{p}{,}  \PY{o}{.}\PY{l+m+mi}{58}\PY{p}{,}  \PY{o}{.}\PY{l+m+mi}{02}\PY{p}{,}  \PY{o}{.}\PY{l+m+mi}{08}\PY{p}{,}    \PY{l+m+mi}{0}\PY{p}{,}    \PY{l+m+mi}{0}\PY{p}{,}    \PY{l+m+mi}{0}\PY{p}{,} \PY{l+m+mi}{0}\PY{p}{]}\PY{p}{,}
                  \PY{p}{[} \PY{l+m+mi}{0}\PY{p}{,}   \PY{o}{.}\PY{l+m+mi}{2}\PY{p}{,}  \PY{o}{.}\PY{l+m+mi}{58}\PY{p}{,}  \PY{l+m+mf}{0.2}\PY{p}{,}  \PY{o}{.}\PY{l+m+mi}{58}\PY{p}{,}  \PY{l+m+mf}{0.1}\PY{p}{,}    \PY{l+m+mi}{0}\PY{p}{,}    \PY{l+m+mi}{0}\PY{p}{,}    \PY{l+m+mi}{0}\PY{p}{,}    \PY{l+m+mi}{0}\PY{p}{,} \PY{l+m+mi}{0}\PY{p}{]}\PY{p}{,}
                  \PY{p}{[} \PY{l+m+mi}{0}\PY{p}{,}    \PY{l+m+mi}{0}\PY{p}{,}    \PY{l+m+mi}{0}\PY{p}{,}  \PY{o}{.}\PY{l+m+mi}{58}\PY{p}{,}   \PY{o}{.}\PY{l+m+mi}{4}\PY{p}{,}    \PY{l+m+mi}{0}\PY{p}{,}    \PY{l+m+mi}{0}\PY{p}{,}    \PY{l+m+mi}{0}\PY{p}{,}    \PY{l+m+mi}{0}\PY{p}{,}    \PY{l+m+mi}{0}\PY{p}{,} \PY{l+m+mi}{0}\PY{p}{]}\PY{p}{,}
                  \PY{p}{[} \PY{l+m+mi}{0}\PY{p}{,}    \PY{l+m+mi}{0}\PY{p}{,}    \PY{l+m+mi}{0}\PY{p}{,}   \PY{o}{.}\PY{l+m+mi}{2}\PY{p}{,}    \PY{l+m+mi}{0}\PY{p}{,}    \PY{l+m+mi}{0}\PY{p}{,}    \PY{l+m+mi}{0}\PY{p}{,}    \PY{l+m+mi}{0}\PY{p}{,}    \PY{l+m+mi}{0}\PY{p}{,}    \PY{l+m+mi}{0}\PY{p}{,} \PY{l+m+mi}{0}\PY{p}{]}\PY{p}{,}
                  \PY{p}{[} \PY{l+m+mi}{0}\PY{p}{,}    \PY{l+m+mi}{0}\PY{p}{,}    \PY{l+m+mi}{0}\PY{p}{,}    \PY{l+m+mi}{0}\PY{p}{,}    \PY{l+m+mi}{0}\PY{p}{,}    \PY{l+m+mi}{0}\PY{p}{,}    \PY{l+m+mi}{0}\PY{p}{,}    \PY{l+m+mi}{0}\PY{p}{,}    \PY{l+m+mi}{0}\PY{p}{,}    \PY{l+m+mi}{0}\PY{p}{,} \PY{l+m+mi}{0}\PY{p}{]}\PY{p}{]}\PY{p}{)}

\PY{c+c1}{\PYZsh{} Figures objects}
\PY{n}{fig}\PY{p}{,} \PY{n}{axis} \PY{o}{=} \PY{n}{plt}\PY{o}{.}\PY{n}{subplots}\PY{p}{(}\PY{n}{nrows}\PY{o}{=}\PY{l+m+mi}{1}\PY{p}{,} \PY{n}{ncols}\PY{o}{=}\PY{l+m+mi}{2}\PY{p}{,} \PY{n}{figsize}\PY{o}{=}\PY{p}{(}\PY{l+m+mi}{14}\PY{p}{,} \PY{l+m+mi}{5}\PY{p}{)}\PY{p}{)}

\PY{c+c1}{\PYZsh{} Image plot}
\PY{n}{gray} \PY{o}{=} \PY{n}{axis}\PY{p}{[}\PY{l+m+mi}{0}\PY{p}{]}\PY{o}{.}\PY{n}{imshow}\PY{p}{(}\PY{n}{image}\PY{p}{,} \PY{n}{cmap}\PY{o}{=}\PY{l+s+s1}{\PYZsq{}}\PY{l+s+s1}{binary}\PY{l+s+s1}{\PYZsq{}}\PY{p}{,} \PY{n}{vmin}\PY{o}{=}\PY{l+m+mi}{0}\PY{p}{,} \PY{n}{vmax}\PY{o}{=}\PY{o}{.}\PY{l+m+mi}{6}\PY{p}{)}
\PY{n}{fig}\PY{o}{.}\PY{n}{colorbar}\PY{p}{(}\PY{n}{gray}\PY{p}{,} \PY{n}{ax}\PY{o}{=}\PY{n}{axis}\PY{p}{[}\PY{l+m+mi}{0}\PY{p}{]}\PY{p}{)}

\PY{c+c1}{\PYZsh{} Histogram plot}
\PY{n}{axis}\PY{p}{[}\PY{l+m+mi}{1}\PY{p}{]}\PY{o}{.}\PY{n}{hist}\PY{p}{(}\PY{n}{image}\PY{o}{.}\PY{n}{flatten}\PY{p}{(}\PY{p}{)}\PY{p}{,}
             \PY{n}{bins}\PY{o}{=}\PY{p}{[}\PY{l+m+mi}{0}\PY{p}{,} \PY{o}{.}\PY{l+m+mi}{05}\PY{p}{,} \PY{o}{.}\PY{l+m+mi}{1}\PY{p}{,} \PY{o}{.}\PY{l+m+mi}{15}\PY{p}{,} \PY{o}{.}\PY{l+m+mi}{2}\PY{p}{,} \PY{o}{.}\PY{l+m+mi}{25}\PY{p}{,} \PY{o}{.}\PY{l+m+mi}{3}\PY{p}{,}
                   \PY{o}{.}\PY{l+m+mi}{35}\PY{p}{,} \PY{o}{.}\PY{l+m+mi}{4}\PY{p}{,} \PY{o}{.}\PY{l+m+mi}{45}\PY{p}{,} \PY{o}{.}\PY{l+m+mi}{5}\PY{p}{,} \PY{o}{.}\PY{l+m+mi}{55}\PY{p}{,} \PY{o}{.}\PY{l+m+mi}{6}\PY{p}{]}\PY{p}{,}
             \PY{n}{align}\PY{o}{=}\PY{l+s+s1}{\PYZsq{}}\PY{l+s+s1}{left}\PY{l+s+s1}{\PYZsq{}}\PY{p}{)}
\PY{n}{plt}\PY{o}{.}\PY{n}{show}\PY{p}{(}\PY{p}{)}
\end{Verbatim}
\end{tcolorbox}

    \begin{center}
    \adjustimage{max size={0.9\linewidth}{0.9\paperheight}}{output_4_0.png}
    \end{center}
    { \hspace*{\fill} \\}
    
    \hypertarget{thresholding}{%
\subsection{Thresholding}\label{thresholding}}

The threshold level is therefore the argument for the minimum value of
the histogram of the image. It is expectable by the authors that the
histogram will have a convex shape. We are implementing by sorting the
histogram. So there may be bins with the same occurrence level. In this
case, the least one will be chosen.

    \begin{tcolorbox}[breakable, size=fbox, boxrule=1pt, pad at break*=1mm,colback=cellbackground, colframe=cellborder]
\prompt{In}{incolor}{4}{\boxspacing}
\begin{Verbatim}[commandchars=\\\{\}]
\PY{c+c1}{\PYZsh{} Histogram}
\PY{n}{hist}\PY{p}{,} \PY{n}{bins} \PY{o}{=} \PY{n}{np}\PY{o}{.}\PY{n}{histogram}\PY{p}{(}\PY{n}{image}\PY{o}{.}\PY{n}{flatten}\PY{p}{(}\PY{p}{)}\PY{p}{,}
                          \PY{n}{bins}\PY{o}{=}\PY{p}{[}\PY{l+m+mi}{0}\PY{p}{,} \PY{o}{.}\PY{l+m+mi}{05}\PY{p}{,} \PY{o}{.}\PY{l+m+mi}{1}\PY{p}{,} \PY{o}{.}\PY{l+m+mi}{15}\PY{p}{,} \PY{o}{.}\PY{l+m+mi}{2}\PY{p}{,} \PY{o}{.}\PY{l+m+mi}{25}\PY{p}{,} \PY{o}{.}\PY{l+m+mi}{3}\PY{p}{,}
                                \PY{o}{.}\PY{l+m+mi}{35}\PY{p}{,} \PY{o}{.}\PY{l+m+mi}{4}\PY{p}{,} \PY{o}{.}\PY{l+m+mi}{45}\PY{p}{,} \PY{o}{.}\PY{l+m+mi}{5}\PY{p}{,} \PY{o}{.}\PY{l+m+mi}{55}\PY{p}{,} \PY{o}{.}\PY{l+m+mi}{6}\PY{p}{]}\PY{p}{)}
\PY{c+c1}{\PYZsh{} Compute the threshold level}
\PY{n}{T} \PY{o}{=} \PY{n}{bins}\PY{p}{[}\PY{n}{np}\PY{o}{.}\PY{n}{argsort}\PY{p}{(}\PY{n}{hist}\PY{p}{)}\PY{p}{[}\PY{l+m+mi}{0}\PY{p}{]}\PY{p}{]}

\PY{c+c1}{\PYZsh{} Apply thresholding}
\PY{n}{image}\PY{p}{[}\PY{n}{image} \PY{o}{\PYZgt{}}\PY{o}{=} \PY{n}{T}\PY{p}{]} \PY{o}{=} \PY{l+m+mf}{1.}
\PY{n}{image}\PY{p}{[}\PY{n}{image} \PY{o}{\PYZlt{}} \PY{n}{T}\PY{p}{]} \PY{o}{=} \PY{l+m+mf}{0.}

\PY{c+c1}{\PYZsh{} Plot the resultant figure}
\PY{n}{plt}\PY{o}{.}\PY{n}{imshow}\PY{p}{(}\PY{n}{image}\PY{p}{,} \PY{n}{cmap}\PY{o}{=}\PY{l+s+s1}{\PYZsq{}}\PY{l+s+s1}{binary}\PY{l+s+s1}{\PYZsq{}}\PY{p}{)}
\PY{n}{plt}\PY{o}{.}\PY{n}{colorbar}\PY{p}{(}\PY{p}{)}
\PY{n}{plt}\PY{o}{.}\PY{n}{show}\PY{p}{(}\PY{p}{)}
\end{Verbatim}
\end{tcolorbox}

    \begin{center}
    \adjustimage{max size={0.9\linewidth}{0.9\paperheight}}{output_6_0.png}
    \end{center}
    { \hspace*{\fill} \\}
    
    The same figure as in 2(c) is obtained

\hypertarget{noise-filtering}{%
\subsection{Noise Filtering}\label{noise-filtering}}

This step is defined as follows in the paper:

"\emph{Successively, a noise filtering is performed to eliminate some
artifacts. The scatteres' support is clearly define by means of the
following transformation:}

\[ \tau_{dn} = \begin{cases} \tau_0, ~\mathrm{if}~ N_0 > N_{max}, \\ \tau_{max}, ~\mathrm{if}~ N_0 \leq N_{max} \\ \tau_{th}(x_{n(r)}, y_{n(r)}), ~\mathrm{elsewhere}~ \end{cases} \]

\emph{where (\(x_j, y_j\)) indicates a neighboring position, \(N_p\) is
the dimension of the complete neighborhood system of the sub-domain
located at (\(x_{n(r)}\), \(y_{n(r)}\)), \(N_0\) indicates the number of
pixels of the neighborhood for which \(\tau_{th}(x_j, y_j) = \tau_0\),
and \(N_{max}\) indicates the number of pixels of the neighborhood for
which \(\tau_{th}(x_j, y_j) = \tau_{max}\).}"

There are a couple of things which are not clear:

\begin{enumerate}
\def\labelenumi{\arabic{enumi}.}
\tightlist
\item
  When does the \emph{elsewhere} case happen?
\item
  Are the pixels updated at the same time or sequentailly following some
  order?
\end{enumerate}

The result in the figure 2(d) show that the middle points are converted
to black by means of this algorithm. We are going to get this in
different ways. In all of them, we are neglecting the \emph{elsewhere}
case since we do not know when this happen.

\hypertarget{parallel-update}{%
\subsubsection{Parallel Update}\label{parallel-update}}

In this manner, we firstly compute \(N_0\) and \(N_{max}\) for each cell
without applying the rule and then we are apply the rule. The result is
below:

    \begin{tcolorbox}[breakable, size=fbox, boxrule=1pt, pad at break*=1mm,colback=cellbackground, colframe=cellborder]
\prompt{In}{incolor}{5}{\boxspacing}
\begin{Verbatim}[commandchars=\\\{\}]
\PY{n}{NY}\PY{p}{,} \PY{n}{NX} \PY{o}{=} \PY{n}{image}\PY{o}{.}\PY{n}{shape}
\PY{n}{image2} \PY{o}{=} \PY{n}{np}\PY{o}{.}\PY{n}{copy}\PY{p}{(}\PY{n}{image}\PY{p}{)}

\PY{c+c1}{\PYZsh{} Compute N0 and NMAX for each cell}
\PY{n}{N0}\PY{p}{,} \PY{n}{NMAX} \PY{o}{=} \PY{n}{np}\PY{o}{.}\PY{n}{zeros}\PY{p}{(}\PY{p}{(}\PY{n}{NY}\PY{p}{,} \PY{n}{NX}\PY{p}{)}\PY{p}{,} \PY{n}{dtype}\PY{o}{=}\PY{n+nb}{int}\PY{p}{)}\PY{p}{,} \PY{n}{np}\PY{o}{.}\PY{n}{zeros}\PY{p}{(}\PY{p}{(}\PY{n}{NY}\PY{p}{,} \PY{n}{NX}\PY{p}{)}\PY{p}{,} \PY{n}{dtype}\PY{o}{=}\PY{n+nb}{int}\PY{p}{)}
\PY{k}{for} \PY{n}{i} \PY{o+ow}{in} \PY{n+nb}{range}\PY{p}{(}\PY{n}{NX}\PY{p}{)}\PY{p}{:}
    \PY{k}{for} \PY{n}{j} \PY{o+ow}{in} \PY{n+nb}{range}\PY{p}{(}\PY{n}{NY}\PY{p}{)}\PY{p}{:}
        \PY{n}{m}\PY{p}{,} \PY{n}{n} \PY{o}{=} \PY{n}{neighbors}\PY{p}{(}\PY{n}{i}\PY{p}{,} \PY{n}{j}\PY{p}{,} \PY{n}{NX}\PY{p}{,} \PY{n}{NY}\PY{p}{)}
        \PY{k}{for} \PY{n}{p} \PY{o+ow}{in} \PY{n+nb}{range}\PY{p}{(}\PY{n+nb}{len}\PY{p}{(}\PY{n}{m}\PY{p}{)}\PY{p}{)}\PY{p}{:}
            \PY{k}{if} \PY{n}{image2}\PY{p}{[}\PY{n}{n}\PY{p}{[}\PY{n}{p}\PY{p}{]}\PY{p}{,} \PY{n}{m}\PY{p}{[}\PY{n}{p}\PY{p}{]}\PY{p}{]} \PY{o}{==} \PY{l+m+mi}{0}\PY{p}{:}
                \PY{n}{N0}\PY{p}{[}\PY{n}{j}\PY{p}{,} \PY{n}{i}\PY{p}{]} \PY{o}{+}\PY{o}{=} \PY{l+m+mi}{1}
            \PY{k}{else}\PY{p}{:}
                \PY{n}{NMAX}\PY{p}{[}\PY{n}{j}\PY{p}{,} \PY{n}{i}\PY{p}{]} \PY{o}{+}\PY{o}{=} \PY{l+m+mi}{1}

\PY{c+c1}{\PYZsh{} Apply the rule}
\PY{n}{image2}\PY{p}{[}\PY{n}{N0} \PY{o}{\PYZgt{}} \PY{n}{NMAX}\PY{p}{]} \PY{o}{=} \PY{l+m+mi}{0}
\PY{n}{image2}\PY{p}{[}\PY{n}{N0} \PY{o}{\PYZlt{}}\PY{o}{=} \PY{n}{NMAX}\PY{p}{]} \PY{o}{=} \PY{l+m+mi}{1}

\PY{n}{plt}\PY{o}{.}\PY{n}{imshow}\PY{p}{(}\PY{n}{image2}\PY{p}{,} \PY{n}{cmap}\PY{o}{=}\PY{l+s+s1}{\PYZsq{}}\PY{l+s+s1}{binary}\PY{l+s+s1}{\PYZsq{}}\PY{p}{)}
\PY{n}{plt}\PY{o}{.}\PY{n}{colorbar}\PY{p}{(}\PY{p}{)}
\PY{n}{plt}\PY{o}{.}\PY{n}{show}\PY{p}{(}\PY{p}{)}
\end{Verbatim}
\end{tcolorbox}

    \begin{center}
    \adjustimage{max size={0.9\linewidth}{0.9\paperheight}}{output_8_0.png}
    \end{center}
    { \hspace*{\fill} \\}
    
    \hypertarget{sequential-update}{%
\subsubsection{Sequential update}\label{sequential-update}}

Another way is to update sequentially the cells. Starting from top to
bottom, left to right, the cells are updated sequentially. This may give
a different result.

    \begin{tcolorbox}[breakable, size=fbox, boxrule=1pt, pad at break*=1mm,colback=cellbackground, colframe=cellborder]
\prompt{In}{incolor}{6}{\boxspacing}
\begin{Verbatim}[commandchars=\\\{\}]
\PY{n}{image3} \PY{o}{=} \PY{n}{np}\PY{o}{.}\PY{n}{copy}\PY{p}{(}\PY{n}{image}\PY{p}{)}

\PY{k}{for} \PY{n}{i} \PY{o+ow}{in} \PY{n+nb}{range}\PY{p}{(}\PY{n}{NX}\PY{p}{)}\PY{p}{:}
    \PY{k}{for} \PY{n}{j} \PY{o+ow}{in} \PY{n+nb}{range}\PY{p}{(}\PY{n}{NY}\PY{p}{)}\PY{p}{:}
        \PY{n}{N0}\PY{p}{,} \PY{n}{NMAX} \PY{o}{=} \PY{l+m+mi}{0}\PY{p}{,} \PY{l+m+mi}{0}
        \PY{n}{m}\PY{p}{,} \PY{n}{n} \PY{o}{=} \PY{n}{neighbors}\PY{p}{(}\PY{n}{i}\PY{p}{,} \PY{n}{j}\PY{p}{,} \PY{n}{NX}\PY{p}{,} \PY{n}{NY}\PY{p}{)}
        \PY{k}{for} \PY{n}{p} \PY{o+ow}{in} \PY{n+nb}{range}\PY{p}{(}\PY{n+nb}{len}\PY{p}{(}\PY{n}{m}\PY{p}{)}\PY{p}{)}\PY{p}{:}
            \PY{k}{if} \PY{n}{image3}\PY{p}{[}\PY{n}{n}\PY{p}{[}\PY{n}{p}\PY{p}{]}\PY{p}{,} \PY{n}{m}\PY{p}{[}\PY{n}{p}\PY{p}{]}\PY{p}{]} \PY{o}{==} \PY{l+m+mi}{0}\PY{p}{:}
                \PY{n}{N0} \PY{o}{+}\PY{o}{=} \PY{l+m+mi}{1}
            \PY{k}{else}\PY{p}{:}
                \PY{n}{NMAX} \PY{o}{+}\PY{o}{=} \PY{l+m+mi}{1}
        \PY{k}{if} \PY{n}{N0} \PY{o}{\PYZgt{}} \PY{n}{NMAX}\PY{p}{:}
            \PY{n}{image3}\PY{p}{[}\PY{n}{j}\PY{p}{,} \PY{n}{i}\PY{p}{]} \PY{o}{=} \PY{l+m+mi}{0}
        \PY{k}{elif} \PY{n}{NMAX} \PY{o}{\PYZgt{}}\PY{o}{=} \PY{n}{N0}\PY{p}{:}
            \PY{n}{image3}\PY{p}{[}\PY{n}{j}\PY{p}{,} \PY{n}{i}\PY{p}{]} \PY{o}{=} \PY{l+m+mi}{1}

\PY{n}{plt}\PY{o}{.}\PY{n}{imshow}\PY{p}{(}\PY{n}{image3}\PY{p}{,} \PY{n}{cmap}\PY{o}{=}\PY{l+s+s1}{\PYZsq{}}\PY{l+s+s1}{binary}\PY{l+s+s1}{\PYZsq{}}\PY{p}{)}
\PY{n}{plt}\PY{o}{.}\PY{n}{colorbar}\PY{p}{(}\PY{p}{)}
\PY{n}{plt}\PY{o}{.}\PY{n}{show}\PY{p}{(}\PY{p}{)}
\end{Verbatim}
\end{tcolorbox}

    \begin{center}
    \adjustimage{max size={0.9\linewidth}{0.9\paperheight}}{output_10_0.png}
    \end{center}
    { \hspace*{\fill} \\}
    
    We may change the order to left-right, top-bottom\ldots{}

    \begin{tcolorbox}[breakable, size=fbox, boxrule=1pt, pad at break*=1mm,colback=cellbackground, colframe=cellborder]
\prompt{In}{incolor}{7}{\boxspacing}
\begin{Verbatim}[commandchars=\\\{\}]
\PY{n}{image4} \PY{o}{=} \PY{n}{np}\PY{o}{.}\PY{n}{copy}\PY{p}{(}\PY{n}{image}\PY{p}{)}

\PY{k}{for} \PY{n}{j} \PY{o+ow}{in} \PY{n+nb}{range}\PY{p}{(}\PY{n}{NY}\PY{p}{)}\PY{p}{:}
    \PY{k}{for} \PY{n}{i} \PY{o+ow}{in} \PY{n+nb}{range}\PY{p}{(}\PY{n}{NX}\PY{p}{)}\PY{p}{:}
        \PY{n}{N0}\PY{p}{,} \PY{n}{NMAX} \PY{o}{=} \PY{l+m+mi}{0}\PY{p}{,} \PY{l+m+mi}{0}
        \PY{n}{m}\PY{p}{,} \PY{n}{n} \PY{o}{=} \PY{n}{neighbors}\PY{p}{(}\PY{n}{i}\PY{p}{,} \PY{n}{j}\PY{p}{,} \PY{n}{NX}\PY{p}{,} \PY{n}{NY}\PY{p}{)}
        \PY{k}{for} \PY{n}{p} \PY{o+ow}{in} \PY{n+nb}{range}\PY{p}{(}\PY{n+nb}{len}\PY{p}{(}\PY{n}{m}\PY{p}{)}\PY{p}{)}\PY{p}{:}
            \PY{k}{if} \PY{n}{image4}\PY{p}{[}\PY{n}{n}\PY{p}{[}\PY{n}{p}\PY{p}{]}\PY{p}{,} \PY{n}{m}\PY{p}{[}\PY{n}{p}\PY{p}{]}\PY{p}{]} \PY{o}{==} \PY{l+m+mi}{0}\PY{p}{:}
                \PY{n}{N0} \PY{o}{+}\PY{o}{=} \PY{l+m+mi}{1}
            \PY{k}{else}\PY{p}{:}
                \PY{n}{NMAX} \PY{o}{+}\PY{o}{=} \PY{l+m+mi}{1}
        \PY{k}{if} \PY{n}{N0} \PY{o}{\PYZgt{}} \PY{n}{NMAX}\PY{p}{:}
            \PY{n}{image4}\PY{p}{[}\PY{n}{j}\PY{p}{,} \PY{n}{i}\PY{p}{]} \PY{o}{=} \PY{l+m+mi}{0}
        \PY{k}{elif} \PY{n}{NMAX} \PY{o}{\PYZgt{}}\PY{o}{=} \PY{n}{N0}\PY{p}{:}
            \PY{n}{image4}\PY{p}{[}\PY{n}{j}\PY{p}{,} \PY{n}{i}\PY{p}{]} \PY{o}{=} \PY{l+m+mi}{1}

\PY{n}{plt}\PY{o}{.}\PY{n}{imshow}\PY{p}{(}\PY{n}{image4}\PY{p}{,} \PY{n}{cmap}\PY{o}{=}\PY{l+s+s1}{\PYZsq{}}\PY{l+s+s1}{binary}\PY{l+s+s1}{\PYZsq{}}\PY{p}{)}
\PY{n}{plt}\PY{o}{.}\PY{n}{colorbar}\PY{p}{(}\PY{p}{)}
\PY{n}{plt}\PY{o}{.}\PY{n}{show}\PY{p}{(}\PY{p}{)}
\end{Verbatim}
\end{tcolorbox}

    \begin{center}
    \adjustimage{max size={0.9\linewidth}{0.9\paperheight}}{output_12_0.png}
    \end{center}
    { \hspace*{\fill} \\}
    
    \ldots right-left, top-bottom, \ldots{}

    \begin{tcolorbox}[breakable, size=fbox, boxrule=1pt, pad at break*=1mm,colback=cellbackground, colframe=cellborder]
\prompt{In}{incolor}{8}{\boxspacing}
\begin{Verbatim}[commandchars=\\\{\}]
\PY{n}{image5} \PY{o}{=} \PY{n}{np}\PY{o}{.}\PY{n}{copy}\PY{p}{(}\PY{n}{image}\PY{p}{)}

\PY{k}{for} \PY{n}{j} \PY{o+ow}{in} \PY{n+nb}{range}\PY{p}{(}\PY{n}{NY}\PY{p}{)}\PY{p}{:}
    \PY{k}{for} \PY{n}{i} \PY{o+ow}{in} \PY{n+nb}{range}\PY{p}{(}\PY{n}{NX}\PY{o}{\PYZhy{}}\PY{l+m+mi}{1}\PY{p}{,} \PY{o}{\PYZhy{}}\PY{l+m+mi}{1}\PY{p}{,} \PY{o}{\PYZhy{}}\PY{l+m+mi}{1}\PY{p}{)}\PY{p}{:}
        \PY{n}{N0}\PY{p}{,} \PY{n}{NMAX} \PY{o}{=} \PY{l+m+mi}{0}\PY{p}{,} \PY{l+m+mi}{0}
        \PY{n}{m}\PY{p}{,} \PY{n}{n} \PY{o}{=} \PY{n}{neighbors}\PY{p}{(}\PY{n}{i}\PY{p}{,} \PY{n}{j}\PY{p}{,} \PY{n}{NX}\PY{p}{,} \PY{n}{NY}\PY{p}{)}
        \PY{k}{for} \PY{n}{p} \PY{o+ow}{in} \PY{n+nb}{range}\PY{p}{(}\PY{n+nb}{len}\PY{p}{(}\PY{n}{m}\PY{p}{)}\PY{p}{)}\PY{p}{:}
            \PY{k}{if} \PY{n}{image5}\PY{p}{[}\PY{n}{n}\PY{p}{[}\PY{n}{p}\PY{p}{]}\PY{p}{,} \PY{n}{m}\PY{p}{[}\PY{n}{p}\PY{p}{]}\PY{p}{]} \PY{o}{==} \PY{l+m+mi}{0}\PY{p}{:}
                \PY{n}{N0} \PY{o}{+}\PY{o}{=} \PY{l+m+mi}{1}
            \PY{k}{else}\PY{p}{:}
                \PY{n}{NMAX} \PY{o}{+}\PY{o}{=} \PY{l+m+mi}{1}
        \PY{k}{if} \PY{n}{N0} \PY{o}{\PYZgt{}} \PY{n}{NMAX}\PY{p}{:}
            \PY{n}{image5}\PY{p}{[}\PY{n}{j}\PY{p}{,} \PY{n}{i}\PY{p}{]} \PY{o}{=} \PY{l+m+mi}{0}
        \PY{k}{elif} \PY{n}{NMAX} \PY{o}{\PYZgt{}}\PY{o}{=} \PY{n}{N0}\PY{p}{:}
            \PY{n}{image5}\PY{p}{[}\PY{n}{j}\PY{p}{,} \PY{n}{i}\PY{p}{]} \PY{o}{=} \PY{l+m+mi}{1}

\PY{n}{plt}\PY{o}{.}\PY{n}{imshow}\PY{p}{(}\PY{n}{image5}\PY{p}{,} \PY{n}{cmap}\PY{o}{=}\PY{l+s+s1}{\PYZsq{}}\PY{l+s+s1}{binary}\PY{l+s+s1}{\PYZsq{}}\PY{p}{,} \PY{n}{vmin}\PY{o}{=}\PY{l+m+mi}{0}\PY{p}{,} \PY{n}{vmax}\PY{o}{=}\PY{o}{.}\PY{l+m+mi}{6}\PY{p}{)}
\PY{n}{plt}\PY{o}{.}\PY{n}{colorbar}\PY{p}{(}\PY{p}{)}
\PY{n}{plt}\PY{o}{.}\PY{n}{show}\PY{p}{(}\PY{p}{)}
\end{Verbatim}
\end{tcolorbox}

    \begin{center}
    \adjustimage{max size={0.9\linewidth}{0.9\paperheight}}{output_14_0.png}
    \end{center}
    { \hspace*{\fill} \\}
    
    \ldots right-left, bottom-top\ldots{}

    \begin{tcolorbox}[breakable, size=fbox, boxrule=1pt, pad at break*=1mm,colback=cellbackground, colframe=cellborder]
\prompt{In}{incolor}{9}{\boxspacing}
\begin{Verbatim}[commandchars=\\\{\}]
\PY{n}{image6} \PY{o}{=} \PY{n}{np}\PY{o}{.}\PY{n}{copy}\PY{p}{(}\PY{n}{image}\PY{p}{)}

\PY{k}{for} \PY{n}{j} \PY{o+ow}{in} \PY{n+nb}{range}\PY{p}{(}\PY{n}{NY}\PY{o}{\PYZhy{}}\PY{l+m+mi}{1}\PY{p}{,} \PY{o}{\PYZhy{}}\PY{l+m+mi}{1}\PY{p}{,} \PY{o}{\PYZhy{}}\PY{l+m+mi}{1}\PY{p}{)}\PY{p}{:}
    \PY{k}{for} \PY{n}{i} \PY{o+ow}{in} \PY{n+nb}{range}\PY{p}{(}\PY{n}{NX}\PY{o}{\PYZhy{}}\PY{l+m+mi}{1}\PY{p}{,} \PY{o}{\PYZhy{}}\PY{l+m+mi}{1}\PY{p}{,} \PY{o}{\PYZhy{}}\PY{l+m+mi}{1}\PY{p}{)}\PY{p}{:}
        \PY{n}{N0}\PY{p}{,} \PY{n}{NMAX} \PY{o}{=} \PY{l+m+mi}{0}\PY{p}{,} \PY{l+m+mi}{0}
        \PY{n}{m}\PY{p}{,} \PY{n}{n} \PY{o}{=} \PY{n}{neighbors}\PY{p}{(}\PY{n}{i}\PY{p}{,} \PY{n}{j}\PY{p}{,} \PY{n}{NX}\PY{p}{,} \PY{n}{NY}\PY{p}{)}
        \PY{k}{for} \PY{n}{p} \PY{o+ow}{in} \PY{n+nb}{range}\PY{p}{(}\PY{n+nb}{len}\PY{p}{(}\PY{n}{m}\PY{p}{)}\PY{p}{)}\PY{p}{:}
            \PY{k}{if} \PY{n}{image6}\PY{p}{[}\PY{n}{n}\PY{p}{[}\PY{n}{p}\PY{p}{]}\PY{p}{,} \PY{n}{m}\PY{p}{[}\PY{n}{p}\PY{p}{]}\PY{p}{]} \PY{o}{==} \PY{l+m+mi}{0}\PY{p}{:}
                \PY{n}{N0} \PY{o}{+}\PY{o}{=} \PY{l+m+mi}{1}
            \PY{k}{else}\PY{p}{:}
                \PY{n}{NMAX} \PY{o}{+}\PY{o}{=} \PY{l+m+mi}{1}
        \PY{k}{if} \PY{n}{N0} \PY{o}{\PYZgt{}} \PY{n}{NMAX}\PY{p}{:}
            \PY{n}{image6}\PY{p}{[}\PY{n}{j}\PY{p}{,} \PY{n}{i}\PY{p}{]} \PY{o}{=} \PY{l+m+mi}{0}
        \PY{k}{elif} \PY{n}{NMAX} \PY{o}{\PYZgt{}}\PY{o}{=} \PY{n}{N0}\PY{p}{:}
            \PY{n}{image6}\PY{p}{[}\PY{n}{j}\PY{p}{,} \PY{n}{i}\PY{p}{]} \PY{o}{=} \PY{l+m+mi}{1}

\PY{n}{plt}\PY{o}{.}\PY{n}{imshow}\PY{p}{(}\PY{n}{image6}\PY{p}{,} \PY{n}{cmap}\PY{o}{=}\PY{l+s+s1}{\PYZsq{}}\PY{l+s+s1}{binary}\PY{l+s+s1}{\PYZsq{}}\PY{p}{,} \PY{n}{vmin}\PY{o}{=}\PY{l+m+mi}{0}\PY{p}{,} \PY{n}{vmax}\PY{o}{=}\PY{o}{.}\PY{l+m+mi}{6}\PY{p}{)}
\PY{n}{plt}\PY{o}{.}\PY{n}{colorbar}\PY{p}{(}\PY{p}{)}
\PY{n}{plt}\PY{o}{.}\PY{n}{show}\PY{p}{(}\PY{p}{)}
\end{Verbatim}
\end{tcolorbox}

    \begin{center}
    \adjustimage{max size={0.9\linewidth}{0.9\paperheight}}{output_16_0.png}
    \end{center}
    { \hspace*{\fill} \\}
    
    \ldots{} bottom-top, right-left\ldots{}

    \begin{tcolorbox}[breakable, size=fbox, boxrule=1pt, pad at break*=1mm,colback=cellbackground, colframe=cellborder]
\prompt{In}{incolor}{10}{\boxspacing}
\begin{Verbatim}[commandchars=\\\{\}]
\PY{n}{image7} \PY{o}{=} \PY{n}{np}\PY{o}{.}\PY{n}{copy}\PY{p}{(}\PY{n}{image}\PY{p}{)}

\PY{k}{for} \PY{n}{i} \PY{o+ow}{in} \PY{n+nb}{range}\PY{p}{(}\PY{n}{NX}\PY{o}{\PYZhy{}}\PY{l+m+mi}{1}\PY{p}{,} \PY{o}{\PYZhy{}}\PY{l+m+mi}{1}\PY{p}{,} \PY{o}{\PYZhy{}}\PY{l+m+mi}{1}\PY{p}{)}\PY{p}{:}
    \PY{k}{for} \PY{n}{j} \PY{o+ow}{in} \PY{n+nb}{range}\PY{p}{(}\PY{n}{NY}\PY{o}{\PYZhy{}}\PY{l+m+mi}{1}\PY{p}{,} \PY{o}{\PYZhy{}}\PY{l+m+mi}{1}\PY{p}{,} \PY{o}{\PYZhy{}}\PY{l+m+mi}{1}\PY{p}{)}\PY{p}{:}
        \PY{n}{N0}\PY{p}{,} \PY{n}{NMAX} \PY{o}{=} \PY{l+m+mi}{0}\PY{p}{,} \PY{l+m+mi}{0}
        \PY{n}{m}\PY{p}{,} \PY{n}{n} \PY{o}{=} \PY{n}{neighbors}\PY{p}{(}\PY{n}{i}\PY{p}{,} \PY{n}{j}\PY{p}{,} \PY{n}{NX}\PY{p}{,} \PY{n}{NY}\PY{p}{)}
        \PY{k}{for} \PY{n}{p} \PY{o+ow}{in} \PY{n+nb}{range}\PY{p}{(}\PY{n+nb}{len}\PY{p}{(}\PY{n}{m}\PY{p}{)}\PY{p}{)}\PY{p}{:}
            \PY{k}{if} \PY{n}{image7}\PY{p}{[}\PY{n}{n}\PY{p}{[}\PY{n}{p}\PY{p}{]}\PY{p}{,} \PY{n}{m}\PY{p}{[}\PY{n}{p}\PY{p}{]}\PY{p}{]} \PY{o}{==} \PY{l+m+mi}{0}\PY{p}{:}
                \PY{n}{N0} \PY{o}{+}\PY{o}{=} \PY{l+m+mi}{1}
            \PY{k}{else}\PY{p}{:}
                \PY{n}{NMAX} \PY{o}{+}\PY{o}{=} \PY{l+m+mi}{1}
        \PY{k}{if} \PY{n}{N0} \PY{o}{\PYZgt{}} \PY{n}{NMAX}\PY{p}{:}
            \PY{n}{image7}\PY{p}{[}\PY{n}{j}\PY{p}{,} \PY{n}{i}\PY{p}{]} \PY{o}{=} \PY{l+m+mi}{0}
        \PY{k}{elif} \PY{n}{NMAX} \PY{o}{\PYZgt{}}\PY{o}{=} \PY{n}{N0}\PY{p}{:}
            \PY{n}{image7}\PY{p}{[}\PY{n}{j}\PY{p}{,} \PY{n}{i}\PY{p}{]} \PY{o}{=} \PY{l+m+mi}{1}

\PY{n}{plt}\PY{o}{.}\PY{n}{imshow}\PY{p}{(}\PY{n}{image7}\PY{p}{,} \PY{n}{cmap}\PY{o}{=}\PY{l+s+s1}{\PYZsq{}}\PY{l+s+s1}{binary}\PY{l+s+s1}{\PYZsq{}}\PY{p}{,} \PY{n}{vmin}\PY{o}{=}\PY{l+m+mi}{0}\PY{p}{,} \PY{n}{vmax}\PY{o}{=}\PY{o}{.}\PY{l+m+mi}{6}\PY{p}{)}
\PY{n}{plt}\PY{o}{.}\PY{n}{colorbar}\PY{p}{(}\PY{p}{)}
\PY{n}{plt}\PY{o}{.}\PY{n}{show}\PY{p}{(}\PY{p}{)}
\end{Verbatim}
\end{tcolorbox}

    \begin{center}
    \adjustimage{max size={0.9\linewidth}{0.9\paperheight}}{output_18_0.png}
    \end{center}
    { \hspace*{\fill} \\}
    
    \hypertarget{only-white-pixels}{%
\subsubsection{Only white pixels}\label{only-white-pixels}}

Another possible manner is to update only white cells, paralely\ldots{}

    \begin{tcolorbox}[breakable, size=fbox, boxrule=1pt, pad at break*=1mm,colback=cellbackground, colframe=cellborder]
\prompt{In}{incolor}{11}{\boxspacing}
\begin{Verbatim}[commandchars=\\\{\}]
\PY{n}{image8} \PY{o}{=} \PY{n}{np}\PY{o}{.}\PY{n}{copy}\PY{p}{(}\PY{n}{image}\PY{p}{)}

\PY{c+c1}{\PYZsh{} Compute N0 and NMAX for each cell}
\PY{n}{N0}\PY{p}{,} \PY{n}{NMAX} \PY{o}{=} \PY{n}{np}\PY{o}{.}\PY{n}{zeros}\PY{p}{(}\PY{p}{(}\PY{n}{NY}\PY{p}{,} \PY{n}{NX}\PY{p}{)}\PY{p}{,} \PY{n}{dtype}\PY{o}{=}\PY{n+nb}{int}\PY{p}{)}\PY{p}{,} \PY{n}{np}\PY{o}{.}\PY{n}{zeros}\PY{p}{(}\PY{p}{(}\PY{n}{NY}\PY{p}{,} \PY{n}{NX}\PY{p}{)}\PY{p}{,} \PY{n}{dtype}\PY{o}{=}\PY{n+nb}{int}\PY{p}{)}
\PY{k}{for} \PY{n}{i} \PY{o+ow}{in} \PY{n+nb}{range}\PY{p}{(}\PY{n}{NX}\PY{p}{)}\PY{p}{:}
    \PY{k}{for} \PY{n}{j} \PY{o+ow}{in} \PY{n+nb}{range}\PY{p}{(}\PY{n}{NY}\PY{p}{)}\PY{p}{:}
        \PY{n}{m}\PY{p}{,} \PY{n}{n} \PY{o}{=} \PY{n}{neighbors}\PY{p}{(}\PY{n}{i}\PY{p}{,} \PY{n}{j}\PY{p}{,} \PY{n}{NX}\PY{p}{,} \PY{n}{NY}\PY{p}{)}
        \PY{k}{for} \PY{n}{p} \PY{o+ow}{in} \PY{n+nb}{range}\PY{p}{(}\PY{n+nb}{len}\PY{p}{(}\PY{n}{m}\PY{p}{)}\PY{p}{)}\PY{p}{:}
            \PY{k}{if} \PY{n}{image8}\PY{p}{[}\PY{n}{n}\PY{p}{[}\PY{n}{p}\PY{p}{]}\PY{p}{,} \PY{n}{m}\PY{p}{[}\PY{n}{p}\PY{p}{]}\PY{p}{]} \PY{o}{==} \PY{l+m+mi}{0}\PY{p}{:}
                \PY{n}{N0}\PY{p}{[}\PY{n}{j}\PY{p}{,} \PY{n}{i}\PY{p}{]} \PY{o}{+}\PY{o}{=} \PY{l+m+mi}{1}
            \PY{k}{else}\PY{p}{:}
                \PY{n}{NMAX}\PY{p}{[}\PY{n}{j}\PY{p}{,} \PY{n}{i}\PY{p}{]} \PY{o}{+}\PY{o}{=} \PY{l+m+mi}{1}

\PY{c+c1}{\PYZsh{} Apply the rule}
\PY{n}{image8}\PY{p}{[}\PY{n}{np}\PY{o}{.}\PY{n}{logical\PYZus{}and}\PY{p}{(}\PY{n}{image8} \PY{o}{==} \PY{l+m+mi}{0}\PY{p}{,} \PY{n}{N0} \PY{o}{\PYZlt{}}\PY{o}{=} \PY{n}{NMAX}\PY{p}{)}\PY{p}{]} \PY{o}{=} \PY{l+m+mi}{1}

\PY{n}{plt}\PY{o}{.}\PY{n}{imshow}\PY{p}{(}\PY{n}{image8}\PY{p}{,} \PY{n}{cmap}\PY{o}{=}\PY{l+s+s1}{\PYZsq{}}\PY{l+s+s1}{binary}\PY{l+s+s1}{\PYZsq{}}\PY{p}{)}
\PY{n}{plt}\PY{o}{.}\PY{n}{colorbar}\PY{p}{(}\PY{p}{)}
\PY{n}{plt}\PY{o}{.}\PY{n}{show}\PY{p}{(}\PY{p}{)}
\end{Verbatim}
\end{tcolorbox}

    \begin{center}
    \adjustimage{max size={0.9\linewidth}{0.9\paperheight}}{output_20_0.png}
    \end{center}
    { \hspace*{\fill} \\}
    
    \ldots{} or sequentially\ldots{}

    \begin{tcolorbox}[breakable, size=fbox, boxrule=1pt, pad at break*=1mm,colback=cellbackground, colframe=cellborder]
\prompt{In}{incolor}{12}{\boxspacing}
\begin{Verbatim}[commandchars=\\\{\}]
\PY{n}{image9} \PY{o}{=} \PY{n}{np}\PY{o}{.}\PY{n}{copy}\PY{p}{(}\PY{n}{image}\PY{p}{)}

\PY{c+c1}{\PYZsh{} Compute N0 and NMAX for each cell}
\PY{k}{for} \PY{n}{i} \PY{o+ow}{in} \PY{n+nb}{range}\PY{p}{(}\PY{n}{NX}\PY{p}{)}\PY{p}{:}
    \PY{k}{for} \PY{n}{j} \PY{o+ow}{in} \PY{n+nb}{range}\PY{p}{(}\PY{n}{NY}\PY{p}{)}\PY{p}{:}
        \PY{k}{if} \PY{n}{image9}\PY{p}{[}\PY{n}{j}\PY{p}{,} \PY{n}{i}\PY{p}{]} \PY{o}{==} \PY{l+m+mi}{0}\PY{p}{:}
            \PY{n}{m}\PY{p}{,} \PY{n}{n} \PY{o}{=} \PY{n}{neighbors}\PY{p}{(}\PY{n}{i}\PY{p}{,} \PY{n}{j}\PY{p}{,} \PY{n}{NX}\PY{p}{,} \PY{n}{NY}\PY{p}{)}
            \PY{n}{NMAX}\PY{p}{,} \PY{n}{N0} \PY{o}{=} \PY{l+m+mi}{0}\PY{p}{,} \PY{l+m+mi}{0}
            \PY{k}{for} \PY{n}{p} \PY{o+ow}{in} \PY{n+nb}{range}\PY{p}{(}\PY{n+nb}{len}\PY{p}{(}\PY{n}{m}\PY{p}{)}\PY{p}{)}\PY{p}{:}
                \PY{k}{if} \PY{n}{image9}\PY{p}{[}\PY{n}{n}\PY{p}{[}\PY{n}{p}\PY{p}{]}\PY{p}{,} \PY{n}{m}\PY{p}{[}\PY{n}{p}\PY{p}{]}\PY{p}{]} \PY{o}{==} \PY{l+m+mi}{0}\PY{p}{:}
                    \PY{n}{N0} \PY{o}{+}\PY{o}{=} \PY{l+m+mi}{1}
                \PY{k}{else}\PY{p}{:}
                    \PY{n}{NMAX} \PY{o}{+}\PY{o}{=} \PY{l+m+mi}{1}
            \PY{k}{if} \PY{n}{N0} \PY{o}{\PYZlt{}}\PY{o}{=} \PY{n}{NMAX}\PY{p}{:}
                \PY{n}{image9}\PY{p}{[}\PY{n}{j}\PY{p}{,} \PY{n}{i}\PY{p}{]} \PY{o}{=} \PY{l+m+mi}{1}

\PY{n}{plt}\PY{o}{.}\PY{n}{imshow}\PY{p}{(}\PY{n}{image9}\PY{p}{,} \PY{n}{cmap}\PY{o}{=}\PY{l+s+s1}{\PYZsq{}}\PY{l+s+s1}{binary}\PY{l+s+s1}{\PYZsq{}}\PY{p}{)}
\PY{n}{plt}\PY{o}{.}\PY{n}{colorbar}\PY{p}{(}\PY{p}{)}
\PY{n}{plt}\PY{o}{.}\PY{n}{show}\PY{p}{(}\PY{p}{)}
\end{Verbatim}
\end{tcolorbox}

    \begin{center}
    \adjustimage{max size={0.9\linewidth}{0.9\paperheight}}{output_22_0.png}
    \end{center}
    { \hspace*{\fill} \\}
    
    But what do we want to forget black pixels in the middle of nothing
(which possible means noise)?

\hypertarget{object-detection}{%
\subsection{Object Detection}\label{object-detection}}

The object detection procude is even harder to understant. The authors
say:

``\emph{Finally, the object detection is performed. The binarized image
is raster scanned from the left- to right-hand side and from the top to
bottom. The current pixel \((x_{n(r)}, y_{n(r)})\) is labeled as
belonging to an object or to the background by examining its
connectivity to the right-hand-side neighbors
\((\hat{x}_j, \hat{y}_j)\), \(j = 1,\cdots, N_p^{(rh)}\). For example,
if \(\tau_{dn}(x_{n(r)}, y_{n(r)}) = \tau_{max}\), then it is assigned
to the object \(q\) to which it is connected. A new object-label
\((q+1)\) is assigned when a transition from a background pixel to an
isolated object pixel is detected.}''

This kind of thing will work well for this case:

    \begin{tcolorbox}[breakable, size=fbox, boxrule=1pt, pad at break*=1mm,colback=cellbackground, colframe=cellborder]
\prompt{In}{incolor}{13}{\boxspacing}
\begin{Verbatim}[commandchars=\\\{\}]
\PY{n}{image} \PY{o}{=} \PY{n}{np}\PY{o}{.}\PY{n}{copy}\PY{p}{(}\PY{n}{image9}\PY{p}{)}
\PY{n}{labels} \PY{o}{=} \PY{n}{np}\PY{o}{.}\PY{n}{zeros}\PY{p}{(}\PY{n}{image}\PY{o}{.}\PY{n}{shape}\PY{p}{,} \PY{n}{dtype}\PY{o}{=}\PY{n+nb}{int}\PY{p}{)}

\PY{n}{q} \PY{o}{=} \PY{l+m+mi}{0}
\PY{k}{for} \PY{n}{j} \PY{o+ow}{in} \PY{n+nb}{range}\PY{p}{(}\PY{n}{NY}\PY{p}{)}\PY{p}{:}
    \PY{k}{for} \PY{n}{i} \PY{o+ow}{in} \PY{n+nb}{range}\PY{p}{(}\PY{n}{NX}\PY{p}{)}\PY{p}{:}

        \PY{c+c1}{\PYZsh{} If there this pixel is an object}
        \PY{k}{if} \PY{n}{image}\PY{p}{[}\PY{n}{j}\PY{p}{,} \PY{n}{i}\PY{p}{]} \PY{o}{==} \PY{l+m+mi}{1}\PY{p}{:}
            \PY{n}{m}\PY{p}{,} \PY{n}{n} \PY{o}{=} \PY{n}{neighbors}\PY{p}{(}\PY{n}{i}\PY{p}{,} \PY{n}{j}\PY{p}{,} \PY{n}{NX}\PY{p}{,} \PY{n}{NY}\PY{p}{)}
            \PY{k}{for} \PY{n}{k} \PY{o+ow}{in} \PY{n+nb}{range}\PY{p}{(}\PY{n+nb}{len}\PY{p}{(}\PY{n}{m}\PY{p}{)}\PY{p}{)}\PY{p}{:}

                \PY{c+c1}{\PYZsh{} If a neighbor has already been labeled }
                \PY{k}{if} \PY{n}{labels}\PY{p}{[}\PY{n}{n}\PY{p}{[}\PY{n}{k}\PY{p}{]}\PY{p}{,} \PY{n}{m}\PY{p}{[}\PY{n}{k}\PY{p}{]}\PY{p}{]} \PY{o}{!=} \PY{l+m+mi}{0}\PY{p}{:}
                    \PY{n}{labels}\PY{p}{[}\PY{n}{j}\PY{p}{,} \PY{n}{i}\PY{p}{]} \PY{o}{=} \PY{n}{labels}\PY{p}{[}\PY{n}{n}\PY{p}{[}\PY{n}{k}\PY{p}{]}\PY{p}{,} \PY{n}{m}\PY{p}{[}\PY{n}{k}\PY{p}{]}\PY{p}{]}
                    \PY{k}{break}

                \PY{c+c1}{\PYZsh{} New object is detected}
                \PY{k}{elif} \PY{n}{k} \PY{o}{==} \PY{n+nb}{len}\PY{p}{(}\PY{n}{m}\PY{p}{)}\PY{o}{\PYZhy{}}\PY{l+m+mi}{1}\PY{p}{:}
                    \PY{n}{q} \PY{o}{+}\PY{o}{=} \PY{l+m+mi}{1}
                    \PY{n}{labels}\PY{p}{[}\PY{n}{j}\PY{p}{,} \PY{n}{i}\PY{p}{]} \PY{o}{=} \PY{n}{q}

\PY{c+c1}{\PYZsh{} Figures objects}
\PY{n}{fig}\PY{p}{,} \PY{n}{axis} \PY{o}{=} \PY{n}{plt}\PY{o}{.}\PY{n}{subplots}\PY{p}{(}\PY{n}{nrows}\PY{o}{=}\PY{l+m+mi}{1}\PY{p}{,} \PY{n}{ncols}\PY{o}{=}\PY{l+m+mi}{2}\PY{p}{,} \PY{n}{figsize}\PY{o}{=}\PY{p}{(}\PY{l+m+mi}{14}\PY{p}{,} \PY{l+m+mi}{5}\PY{p}{)}\PY{p}{)}

\PY{c+c1}{\PYZsh{} Image plot}
\PY{n}{gray} \PY{o}{=} \PY{n}{axis}\PY{p}{[}\PY{l+m+mi}{0}\PY{p}{]}\PY{o}{.}\PY{n}{imshow}\PY{p}{(}\PY{n}{image}\PY{p}{,} \PY{n}{cmap}\PY{o}{=}\PY{l+s+s1}{\PYZsq{}}\PY{l+s+s1}{binary}\PY{l+s+s1}{\PYZsq{}}\PY{p}{)}
\PY{n}{fig}\PY{o}{.}\PY{n}{colorbar}\PY{p}{(}\PY{n}{gray}\PY{p}{,} \PY{n}{ax}\PY{o}{=}\PY{n}{axis}\PY{p}{[}\PY{l+m+mi}{0}\PY{p}{]}\PY{p}{)}

\PY{c+c1}{\PYZsh{} Label plot}
\PY{n}{color} \PY{o}{=} \PY{n}{axis}\PY{p}{[}\PY{l+m+mi}{1}\PY{p}{]}\PY{o}{.}\PY{n}{imshow}\PY{p}{(}\PY{n}{labels}\PY{p}{)}
\PY{n}{fig}\PY{o}{.}\PY{n}{colorbar}\PY{p}{(}\PY{n}{color}\PY{p}{,} \PY{n}{ax}\PY{o}{=}\PY{n}{axis}\PY{p}{[}\PY{l+m+mi}{1}\PY{p}{]}\PY{p}{)}

\PY{n}{plt}\PY{o}{.}\PY{n}{show}\PY{p}{(}\PY{p}{)}
\end{Verbatim}
\end{tcolorbox}

    \begin{center}
    \adjustimage{max size={0.9\linewidth}{0.9\paperheight}}{output_24_0.png}
    \end{center}
    { \hspace*{\fill} \\}
    
    Now, think for example if we have a ``U'':

    \begin{tcolorbox}[breakable, size=fbox, boxrule=1pt, pad at break*=1mm,colback=cellbackground, colframe=cellborder]
\prompt{In}{incolor}{14}{\boxspacing}
\begin{Verbatim}[commandchars=\\\{\}]
\PY{n}{u} \PY{o}{=} \PY{n}{np}\PY{o}{.}\PY{n}{array}\PY{p}{(}\PY{p}{[}\PY{p}{[} \PY{l+m+mi}{0}\PY{p}{,}    \PY{l+m+mi}{0}\PY{p}{,}    \PY{l+m+mi}{0}\PY{p}{,}    \PY{l+m+mi}{0}\PY{p}{,}    \PY{l+m+mi}{0}\PY{p}{,}    \PY{l+m+mi}{0}\PY{p}{,}    \PY{l+m+mi}{0}\PY{p}{,}    \PY{l+m+mi}{0}\PY{p}{,}    \PY{l+m+mi}{0}\PY{p}{,}    \PY{l+m+mi}{0}\PY{p}{,} \PY{l+m+mi}{0}\PY{p}{]}\PY{p}{,}
              \PY{p}{[} \PY{l+m+mi}{0}\PY{p}{,}    \PY{l+m+mi}{1}\PY{p}{,}    \PY{l+m+mi}{1}\PY{p}{,}    \PY{l+m+mi}{1}\PY{p}{,}    \PY{l+m+mi}{0}\PY{p}{,}    \PY{l+m+mi}{0}\PY{p}{,}    \PY{l+m+mi}{0}\PY{p}{,}    \PY{l+m+mi}{1}\PY{p}{,}    \PY{l+m+mi}{1}\PY{p}{,}    \PY{l+m+mi}{1}\PY{p}{,} \PY{l+m+mi}{0}\PY{p}{]}\PY{p}{,}
              \PY{p}{[} \PY{l+m+mi}{0}\PY{p}{,}    \PY{l+m+mi}{1}\PY{p}{,}    \PY{l+m+mi}{1}\PY{p}{,}    \PY{l+m+mi}{1}\PY{p}{,}    \PY{l+m+mi}{0}\PY{p}{,}    \PY{l+m+mi}{0}\PY{p}{,}    \PY{l+m+mi}{0}\PY{p}{,}    \PY{l+m+mi}{1}\PY{p}{,}    \PY{l+m+mi}{1}\PY{p}{,}    \PY{l+m+mi}{1}\PY{p}{,} \PY{l+m+mi}{0}\PY{p}{]}\PY{p}{,}
              \PY{p}{[} \PY{l+m+mi}{0}\PY{p}{,}    \PY{l+m+mi}{1}\PY{p}{,}    \PY{l+m+mi}{1}\PY{p}{,}    \PY{l+m+mi}{1}\PY{p}{,}    \PY{l+m+mi}{0}\PY{p}{,}    \PY{l+m+mi}{0}\PY{p}{,}    \PY{l+m+mi}{0}\PY{p}{,}    \PY{l+m+mi}{1}\PY{p}{,}    \PY{l+m+mi}{1}\PY{p}{,}    \PY{l+m+mi}{1}\PY{p}{,} \PY{l+m+mi}{0}\PY{p}{]}\PY{p}{,}
              \PY{p}{[} \PY{l+m+mi}{0}\PY{p}{,}    \PY{l+m+mi}{1}\PY{p}{,}    \PY{l+m+mi}{1}\PY{p}{,}    \PY{l+m+mi}{1}\PY{p}{,}    \PY{l+m+mi}{1}\PY{p}{,}    \PY{l+m+mi}{1}\PY{p}{,}    \PY{l+m+mi}{1}\PY{p}{,}    \PY{l+m+mi}{1}\PY{p}{,}    \PY{l+m+mi}{1}\PY{p}{,}    \PY{l+m+mi}{1}\PY{p}{,} \PY{l+m+mi}{0}\PY{p}{]}\PY{p}{,}
              \PY{p}{[} \PY{l+m+mi}{0}\PY{p}{,}    \PY{l+m+mi}{1}\PY{p}{,}    \PY{l+m+mi}{1}\PY{p}{,}    \PY{l+m+mi}{1}\PY{p}{,}    \PY{l+m+mi}{1}\PY{p}{,}    \PY{l+m+mi}{1}\PY{p}{,}    \PY{l+m+mi}{1}\PY{p}{,}    \PY{l+m+mi}{1}\PY{p}{,}    \PY{l+m+mi}{1}\PY{p}{,}    \PY{l+m+mi}{1}\PY{p}{,} \PY{l+m+mi}{0}\PY{p}{]}\PY{p}{,}
              \PY{p}{[} \PY{l+m+mi}{0}\PY{p}{,}    \PY{l+m+mi}{1}\PY{p}{,}    \PY{l+m+mi}{1}\PY{p}{,}    \PY{l+m+mi}{1}\PY{p}{,}    \PY{l+m+mi}{1}\PY{p}{,}    \PY{l+m+mi}{1}\PY{p}{,}    \PY{l+m+mi}{1}\PY{p}{,}    \PY{l+m+mi}{1}\PY{p}{,}    \PY{l+m+mi}{1}\PY{p}{,}    \PY{l+m+mi}{1}\PY{p}{,} \PY{l+m+mi}{0}\PY{p}{]}\PY{p}{,}
              \PY{p}{[} \PY{l+m+mi}{0}\PY{p}{,}    \PY{l+m+mi}{0}\PY{p}{,}    \PY{l+m+mi}{0}\PY{p}{,}    \PY{l+m+mi}{0}\PY{p}{,}    \PY{l+m+mi}{0}\PY{p}{,}    \PY{l+m+mi}{0}\PY{p}{,}    \PY{l+m+mi}{0}\PY{p}{,}    \PY{l+m+mi}{0}\PY{p}{,}    \PY{l+m+mi}{0}\PY{p}{,}    \PY{l+m+mi}{0}\PY{p}{,} \PY{l+m+mi}{0}\PY{p}{]}\PY{p}{,}
              \PY{p}{[} \PY{l+m+mi}{0}\PY{p}{,}    \PY{l+m+mi}{0}\PY{p}{,}    \PY{l+m+mi}{0}\PY{p}{,}    \PY{l+m+mi}{0}\PY{p}{,}    \PY{l+m+mi}{0}\PY{p}{,}    \PY{l+m+mi}{0}\PY{p}{,}    \PY{l+m+mi}{0}\PY{p}{,}    \PY{l+m+mi}{0}\PY{p}{,}    \PY{l+m+mi}{0}\PY{p}{,}    \PY{l+m+mi}{0}\PY{p}{,} \PY{l+m+mi}{0}\PY{p}{]}\PY{p}{,}
              \PY{p}{[} \PY{l+m+mi}{0}\PY{p}{,}    \PY{l+m+mi}{0}\PY{p}{,}    \PY{l+m+mi}{0}\PY{p}{,}    \PY{l+m+mi}{0}\PY{p}{,}    \PY{l+m+mi}{0}\PY{p}{,}    \PY{l+m+mi}{0}\PY{p}{,}    \PY{l+m+mi}{0}\PY{p}{,}    \PY{l+m+mi}{0}\PY{p}{,}    \PY{l+m+mi}{0}\PY{p}{,}    \PY{l+m+mi}{0}\PY{p}{,} \PY{l+m+mi}{0}\PY{p}{]}\PY{p}{,}
              \PY{p}{[} \PY{l+m+mi}{0}\PY{p}{,}    \PY{l+m+mi}{0}\PY{p}{,}    \PY{l+m+mi}{0}\PY{p}{,}    \PY{l+m+mi}{0}\PY{p}{,}    \PY{l+m+mi}{0}\PY{p}{,}    \PY{l+m+mi}{0}\PY{p}{,}    \PY{l+m+mi}{0}\PY{p}{,}    \PY{l+m+mi}{0}\PY{p}{,}    \PY{l+m+mi}{0}\PY{p}{,}    \PY{l+m+mi}{0}\PY{p}{,} \PY{l+m+mi}{0}\PY{p}{]}\PY{p}{]}\PY{p}{)}

\PY{n}{plt}\PY{o}{.}\PY{n}{imshow}\PY{p}{(}\PY{n}{u}\PY{p}{,} \PY{n}{cmap}\PY{o}{=}\PY{l+s+s1}{\PYZsq{}}\PY{l+s+s1}{binary}\PY{l+s+s1}{\PYZsq{}}\PY{p}{)}
\PY{n}{plt}\PY{o}{.}\PY{n}{show}\PY{p}{(}\PY{p}{)}
\end{Verbatim}
\end{tcolorbox}

    \begin{center}
    \adjustimage{max size={0.9\linewidth}{0.9\paperheight}}{output_26_0.png}
    \end{center}
    { \hspace*{\fill} \\}
    
    If we apply the same algorithm:

    \begin{tcolorbox}[breakable, size=fbox, boxrule=1pt, pad at break*=1mm,colback=cellbackground, colframe=cellborder]
\prompt{In}{incolor}{21}{\boxspacing}
\begin{Verbatim}[commandchars=\\\{\}]
\PY{n}{labels} \PY{o}{=} \PY{n}{np}\PY{o}{.}\PY{n}{zeros}\PY{p}{(}\PY{n}{u}\PY{o}{.}\PY{n}{shape}\PY{p}{,} \PY{n}{dtype}\PY{o}{=}\PY{n+nb}{int}\PY{p}{)}

\PY{n}{q} \PY{o}{=} \PY{l+m+mi}{0}
\PY{k}{for} \PY{n}{j} \PY{o+ow}{in} \PY{n+nb}{range}\PY{p}{(}\PY{n}{NY}\PY{p}{)}\PY{p}{:}
    \PY{k}{for} \PY{n}{i} \PY{o+ow}{in} \PY{n+nb}{range}\PY{p}{(}\PY{n}{NX}\PY{p}{)}\PY{p}{:}
    
        \PY{c+c1}{\PYZsh{} If there this pixel is an object}
        \PY{k}{if} \PY{n}{u}\PY{p}{[}\PY{n}{j}\PY{p}{,} \PY{n}{i}\PY{p}{]} \PY{o}{==} \PY{l+m+mi}{1}\PY{p}{:}
            \PY{n}{m}\PY{p}{,} \PY{n}{n} \PY{o}{=} \PY{n}{neighbors}\PY{p}{(}\PY{n}{i}\PY{p}{,} \PY{n}{j}\PY{p}{,} \PY{n}{NX}\PY{p}{,} \PY{n}{NY}\PY{p}{)}
            \PY{n}{aux} \PY{o}{=} \PY{n}{np}\PY{o}{.}\PY{n}{zeros}\PY{p}{(}\PY{n+nb}{len}\PY{p}{(}\PY{n}{m}\PY{p}{)}\PY{p}{)}
            \PY{k}{for} \PY{n}{k} \PY{o+ow}{in} \PY{n+nb}{range}\PY{p}{(}\PY{n+nb}{len}\PY{p}{(}\PY{n}{m}\PY{p}{)}\PY{p}{)}\PY{p}{:}

                \PY{c+c1}{\PYZsh{} If a neighbor has already been labeled }
                \PY{k}{if} \PY{n}{labels}\PY{p}{[}\PY{n}{n}\PY{p}{[}\PY{n}{k}\PY{p}{]}\PY{p}{,} \PY{n}{m}\PY{p}{[}\PY{n}{k}\PY{p}{]}\PY{p}{]} \PY{o}{!=} \PY{l+m+mi}{0}\PY{p}{:}
                    \PY{n}{labels}\PY{p}{[}\PY{n}{j}\PY{p}{,} \PY{n}{i}\PY{p}{]} \PY{o}{=} \PY{n}{labels}\PY{p}{[}\PY{n}{n}\PY{p}{[}\PY{n}{k}\PY{p}{]}\PY{p}{,} \PY{n}{m}\PY{p}{[}\PY{n}{k}\PY{p}{]}\PY{p}{]}
                    \PY{k}{break}

                \PY{c+c1}{\PYZsh{} New object is detected}
                \PY{k}{elif} \PY{n}{k} \PY{o}{==} \PY{n+nb}{len}\PY{p}{(}\PY{n}{m}\PY{p}{)}\PY{o}{\PYZhy{}}\PY{l+m+mi}{1}\PY{p}{:}
                    \PY{n}{q} \PY{o}{+}\PY{o}{=} \PY{l+m+mi}{1}
                    \PY{n}{labels}\PY{p}{[}\PY{n}{j}\PY{p}{,} \PY{n}{i}\PY{p}{]} \PY{o}{=} \PY{n}{q}


\PY{c+c1}{\PYZsh{} Figures objects}
\PY{n}{fig}\PY{p}{,} \PY{n}{axis} \PY{o}{=} \PY{n}{plt}\PY{o}{.}\PY{n}{subplots}\PY{p}{(}\PY{n}{nrows}\PY{o}{=}\PY{l+m+mi}{1}\PY{p}{,} \PY{n}{ncols}\PY{o}{=}\PY{l+m+mi}{2}\PY{p}{,} \PY{n}{figsize}\PY{o}{=}\PY{p}{(}\PY{l+m+mi}{14}\PY{p}{,} \PY{l+m+mi}{5}\PY{p}{)}\PY{p}{)}

\PY{c+c1}{\PYZsh{} Image plot}
\PY{n}{gray} \PY{o}{=} \PY{n}{axis}\PY{p}{[}\PY{l+m+mi}{0}\PY{p}{]}\PY{o}{.}\PY{n}{imshow}\PY{p}{(}\PY{n}{u}\PY{p}{,} \PY{n}{cmap}\PY{o}{=}\PY{l+s+s1}{\PYZsq{}}\PY{l+s+s1}{binary}\PY{l+s+s1}{\PYZsq{}}\PY{p}{)}
\PY{n}{fig}\PY{o}{.}\PY{n}{colorbar}\PY{p}{(}\PY{n}{gray}\PY{p}{,} \PY{n}{ax}\PY{o}{=}\PY{n}{axis}\PY{p}{[}\PY{l+m+mi}{0}\PY{p}{]}\PY{p}{)}

\PY{c+c1}{\PYZsh{} Label plot}
\PY{n}{color} \PY{o}{=} \PY{n}{axis}\PY{p}{[}\PY{l+m+mi}{1}\PY{p}{]}\PY{o}{.}\PY{n}{imshow}\PY{p}{(}\PY{n}{labels}\PY{p}{)}
\PY{n}{fig}\PY{o}{.}\PY{n}{colorbar}\PY{p}{(}\PY{n}{color}\PY{p}{,} \PY{n}{ax}\PY{o}{=}\PY{n}{axis}\PY{p}{[}\PY{l+m+mi}{1}\PY{p}{]}\PY{p}{)}

\PY{n}{plt}\PY{o}{.}\PY{n}{show}\PY{p}{(}\PY{p}{)}
\end{Verbatim}
\end{tcolorbox}

    \begin{center}
    \adjustimage{max size={0.9\linewidth}{0.9\paperheight}}{output_28_0.png}
    \end{center}
    { \hspace*{\fill} \\}
    
    In these cases, the zoom may be applied and two areas which overlap are
returned as two different areas to solve the equation. Then, PSO will
have two possible values to represent the same pixel.


    % Add a bibliography block to the postdoc
    
    
    
\end{document}
