% Este é um arquivo para ajuste de configurações do documento. Aqui você pode definir as configurações de pacotes, comandos personalizados, estilos de página, etc.

% Configuração de cabeçalhos e rodapés
\pagestyle{fancy}
\fancyhf{} % Limpa todos os cabeçalhos e rodapés predefinidos
\fancyhead[R]{\thepage} % Número da página no cabeçalho direito
\fancypagestyle{plain}{%
    \fancyhf{}%
    \fancyhead[R]{\thepage}%
}
\renewcommand{\headrulewidth}{0pt} % Remove a linha horizontal do cabeçalho em todos os estilos

\onehalfspacing % Define o espaçamento entre linhas para 1,5
\geometry{left=3cm, top=3cm, bottom=2cm, right=2cm} % Define as margens do documento: esquerda 3 cm, topo 3 cm, inferior 2 cm, direita 2 cm
\setlength{\parindent}{1.25cm} % Define a indentação do parágrafo para 1,25 cm
\setlength{\parskip}{0cm} % Define o espaçamento entre parágrafos para 0 cm
\renewcommand\contentsname{Sumário} % Renomeia o título do sumário para "Sumário"

% Define novos ambientes para teoremas, lemas, proposições, corolários, resultados e definições. Você deve traduzir para o inglês se estiver escrevendo em inglês.
\newtheorem{theorem}{Teorema} % Define o ambiente "theorem" com a numeração independente
\newtheorem{lemma}[theorem]{Lema} % Define o ambiente "lemma" com a mesma numeração do ambiente "theorem"
\newtheorem{proposition}[theorem]{Proposição} % Define o ambiente "proposition" com a mesma numeração do ambiente "theorem"
\newtheorem{corollary}[theorem]{Corolário} % Define o ambiente "corollary" com a mesma numeração do ambiente "theorem"
\newtheorem{result}[theorem]{Resultado} % Define o ambiente "result" com a mesma numeração do ambiente "theorem"
\newtheorem{definition}[theorem]{Definição} % Define o ambiente "definition" com a mesma numeração do ambiente "theorem"

% Definir o comando para armazenar o subtítulo
\newcommand{\thesubtitle}{}
\newcommand{\subtitle}[1]{\renewcommand{\thesubtitle}{#1}}

% Exemplo de comando personalizado
\newcommand{\brho}{\boldsymbol{\rho}}
\newcommand{\brhop}{\boldsymbol{\rho^\prime}}